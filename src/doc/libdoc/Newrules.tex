\documentclass[12pt]{article}
\usepackage{url}
\begin{document}
\begin{enumerate}
\item
Replace the modules for managing resources by threading a
resource structure through the code.
\item
Use your own functions to write error messages and debug messages.
Use the gcc to check the correctness of the format string, see also
E-mail von rhomann, 5.102004
\item
Use basic object oriented style as suggested by Gordon.
\item
Always wrapup the data structures, even in cases where an error occurs.
\item
Never use printf as an output, but a callback function.
\item
Implement a streaming approach to generate the output.
\item
Use variable number of arguments for debug, see
\url{http://gcc.gnu.org/onlinedocs/gcc-4.0.0/gcc/Variadic-Macros.html#Variadic-Macros}
\item
Use \texttt{typeof}-Operator to sometimes declare variables, for example
\begin{verbatim}
typeof (*x) y;
\end{verbatim}
declares a pointer to variables of the type of variable \texttt{x}.
See \url{http://gcc.gnu.org/onlinedocs/gcc-4.0.0/gcc/Typeof.html#Typeof}.
\end{enumerate}
\item
The part of the program managing resources should push references to all
resources onto a stack. If an error occurs, all resources on the stack
should be freed. This would lead to an appropriate behaviour in the error case.
\item
maybe it is better to use Gordon's style of programming.
\end{enumerate}

\section{The following usefull classes exist in Gordon's Implementation}
\begin{enumerate}
\item
\texttt{Alpha} as replacement for \texttt{Alphabet}. However, mapping
of arbitray alphabets is not supported. One maybe should add functions
to output an entire sequence of symbols. It would also be nice to
have functions, which encode a stream of symbols.
\item
\texttt{Bioseq} as replacement for \texttt{Multiseq}. Need example to
see how it works. What about the performance? Reimplement this on the
base of encodedsequences.
\item
\texttt{Fasta-reader}: Reading fasta files.
\item
\texttt{Dlist}: Double linked list.
\item
\texttt{error}: Error functions.
\item
\texttt{hashtable}: Hash table
\item
\texttt{IO}: input/output
\item
\texttt{Mergesort}: with buffer
\item
\texttt{option}: Option parsing
\item
\texttt{progressbar}: progress bar
\item
\texttt{scorefunction}
\item
\texttt{scorematrix}
\end{enumerate}
\end{document}
