\documentclass[12pt]{article}
\begin{document}
\parindent0pt
\parskip10pt

\section*{A whole genome comparison approach to find pathogen signatures}

goal: identify pathogens and distinguish them from harmless near neighbors

first step: locate signature, i.e. region of the pathogen genome that is
unique (does not occur in the available sequence data)

whole genome approach: compare pathogen against all other microbial genomes 
(``Allmicrobes DB'')

use Vmatch software developed at ZBH to 
search for substrings in pathogen (length \(\geq 18\)) that
have a match in Allmicrobes DB

output all regions in pathogen that have \underline{no} match

this subtracts Allmicrobes DB from pathogen sequence

Vmatch can very efficiently handle large datasizes:

requires less than an hour for \(\approx\) 1 GB Allmicrobes DB and typical 
pathogen (e.g.\ B.\ Anthracis: 5 MB)

Vmatch is now part of the KPATH-System (developed at LLNL)
to compute high-quality signatures

success rates of signatures have improve by about 2 orders of magnitude 
(compared to wet-lab designed signatures)

unique assays are now used in a nationwide detection system for pathogens

\end{document}
