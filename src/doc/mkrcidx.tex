
\begin{Showprogramwithoptionswithoutindex}{\MKRC}{}

\Option{db}{\Showoptionarg{dbfiles}}{
Specify a non empty list of database files separated by white spaces.
Each database file contains sequences in one of the following formats:
Fasta, Genbank, EMBL, and SWISSPROT. The user does not have to 
specify the input format. However, the format of all files has to 
be identical. The sequence must consist of characters over the alphabet
\texttt{a}, \texttt{c}, \texttt{g}, \texttt{t}, or \texttt{u} (in
lower or upper case), or wildcards
\texttt{n}, \texttt{s}, \texttt{y}, \texttt{w}, \texttt{r}, \texttt{k},
\texttt{v}, \texttt{b}, \texttt{d}, \texttt{h}, \texttt{m}. 
White spaces in the input files are ignored. This option is mandatory.
This option is identical with the same option 
of the program \MKV.
}

\Option{indexname}{\Showoptionarg{filepath}}{
Specify \Showoptionarg{filepath} to be the name of the index, later referred
to by \emph{indexname}. This option is mandatory, if more than one database 
file is given.
If there is only one database file, and this option is not given, then 
\emph{indexname} is the basename of the given \Showoptionarg{filepath},
i.e.\ the filename stripped by the directory path where it is stored. 
\Showoptionarg{filepath} can be a complete path.
}

\Option{v}{~~~}{%
Be verbose, that is, give reports about the different steps as well as the
resource requirements of the computation. This option is recommended.
}

\Option{version}{~~~}{
Show the version of the program.
Also report the compilation date and the compilation options.
}

\Option{help}{~~~}{
Show a summary of all options and terminate.
}
\end{Showprogramwithoptionswithoutindex}
