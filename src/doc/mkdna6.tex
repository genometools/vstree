\MKDNASIX is very similar to \MKV. While \MKV can handle sequences
over arbitrary alphabets, \MKDNASIX requires DNA-sequences as input.
It generates two indices, namely:
\begin{itemize}
\item
A flat index \emph{indexname} for the the given DNA sequences.
It mainly consists of the two files 
\(\Indexfile{tis}\) and \(\Indexfile{ois}\). This index
is mainly used for output purpose.
\item
An index \emph{indexname.6fr} for the given DNA sequences
translated in all six reading frames. This is used for
computing the matches.
\end{itemize}
These two indices allow \VM to compute matches on the protein level.

\begin{Showprogramwithoptionswithoutindex}{\MKDNASIX}{}

\Option{db}{\Showoptionarg{dbfiles}}{
Specify a non empty list of database files separated by white spaces.
Each database file contains sequences in one of the following formats:
Fasta, Genbank, EMBL, and SWISSPROT. The user does not have to 
specify the input format. However, the format of all files has to 
be identical. The sequence must consist of characters over the alphabet
\texttt{a}, \texttt{c}, \texttt{g}, \texttt{t}, or \texttt{u} (in
lower or upper case), or wildcards
\texttt{n}, \texttt{s}, \texttt{y}, \texttt{w}, \texttt{r}, \texttt{k},
\texttt{v}, \texttt{b}, \texttt{d}, \texttt{h}, \texttt{m}. 
White spaces in the input files are ignored. This option is mandatory.
This option is identical with the same option 
of the program \MKV.
}

\Option{smap}{\Showoptionarg{mapfile}}{
Specify the file storing the symbol mapping applied
to the amino acid symbols after the six frame translation.
\index{symbol mapping}
If the given \Showoptionarg{mapfile} cannot be found in the directory where
\MKV is run, then all directories specified by the environment variable
\Environmentvariable{MKVTREESMAPDIR}
are searched. If defined correctly, this contains a
list of directory paths separated by colons (\texttt{`:'}). For example, 
if one uses the \Programname{csh} or the \Programname{tcsh}, 
the definition of the environment-variable could look like this:

\begin{LargeOutput}
setenv MKVTREESMAPDIR \SY{34}\$HOME/vstree:/usr/vstree/TRANS\SY{34}
\end{LargeOutput}

For the \Programname{bash} or the \Programname{sh} the 
definition could look like

\begin{LargeOutput}
MKVTREESMAPDIR=\SY{34}\$HOME/vstree:/usr/vstree/TRANS\SY{34}

export MKVTREESMAPDIR
\end{LargeOutput}

Then, if \Showoptionarg{mapfile} is not available in the current directory, 
\MKV searches for \Showoptionarg{mapfile} in the two given directories. It
scans the directory-list from left to right. As soon as it has found the file
it stops. If the file cannot be found, an error message is reported and
the program exits with \index{error code} error code 1. 
\AboutMkdnasixcmd{See the manual of the program Vmatch
(\url{http://www.vmatch.de})}
\AboutVmatchcmd{See Appendix \ref{Symbolmap}}
for a more detailed explanation of the format of the symbol mapping file.
}

\Option{transnum}{\Showoptionarg{$t$}}{
Specify the number of the codon translation table used for
the six frame translation. $t$ must be a number in the range \([1,23]\) except
for 7, 8, 17, 18, 19 and 20. Table \ref{Codontranstables} gives the possible numbers
and their names. The codon translation tables, their numbers, and their
names were taken from \url{ftp://ftp.ncbi.nih.gov/entrez/misc/data/gc.prt}.
If this option is not used, then the standard codon translation
table (number 1) is used.
}

\Option{indexname}{\Showoptionarg{filepath}}{
Specify \Showoptionarg{filepath} to be the name of the index, later referred
to by \emph{indexname}. This option is mandatory, if more than one database 
file is given and if additionally at least one file comprising the 
index is stored (i.e.\ if any of the output options is used). If no 
file from the index is stored, then this option is not allowed. 
If there is only one database file, and this option is not given, then 
\emph{indexname} is the basename of the given \Showoptionarg{filepath},
i.e.\ the filename stripped by the directory path where it is stored. 
The \Showoptionarg{filepath} can be a complete path.
}

\Option{tis}{}{
Store the table \(\TIS\) in file \(\Indexfilesixfr{tis}\).
}

\Option{ois}{}{
Store the table \(\OIS\) in file \(\Indexfilesixfr{ois}\).
}

\Option{v}{~~~}{%
Be verbose, that is, give reports about the different steps as well as the
resource requirements of the computation. This option is recommended.
}

\Option{version}{~~~}{
Show the version of the program.
Also report the compilation date and the compilation options.
}

\Option{help}{~~~}{
Show a summary of all options and terminate.
}

\end{Showprogramwithoptionswithoutindex}

\input{codontrans}
