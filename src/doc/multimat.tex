\documentclass[12pt]{article}
\usepackage{url}
\usepackage{a4wide}
\usepackage{alltt}
\usepackage{xspace}
\usepackage{times}
\usepackage{listings}
\usepackage{bbm}
\usepackage{verbatim}
\usepackage{skaff}
\usepackage{prognames}
\usepackage{optionman}
\newcommand{\Size}[1]{|#1|}
\newcommand{\Subchar}[2]{#1[#2]}
\newcommand{\Substring}[3]{#1[#2..#3]}
\newcommand{\MM}{\texttt{multimat}\xspace}
\newcommand{\SEMM}{\texttt{ramaco}\xspace}
\newcommand{\MEMSPEall}{\texttt{Memspeall.pl}\xspace}
\newcommand{\Ignore}[1]{}
\newcommand{\MultiMatch}[2]{(#1,#2_{0},#2_{1},\ldots,#2_{k-1})}
\newcommand{\MMEM}[0]{\textit{multiMEM}\xspace}
\newcommand{\MEM}[0]{\textit{MEM}\xspace}
\newcommand{\MEMset}[0]{\textit{MEMset}\xspace}
\newcommand{\SingleCM}[0]{p}
\newcommand{\Expect}{\mathbbm{E}}
\newcommand{\PR}{\mathbbm{P}}
\newcommand{\MMevalues}{\mathit{mmevalues}}
\newcommand{\Datatype}[1]{\texttt{\small #1}\index{#1@\texttt{#1}}}
\newcommand{\Selectionfunction}[1]{\texttt{\small #1}\index{#1@\texttt{#1}}}
\newcommand{\Filename}[1]{\texttt{\small #1}\index{#1@\texttt{#1}}}
\newcommand{\Keyword}[1]{\texttt{\small #1}\index{#1@\texttt{#1}}}
\newcommand{\Environmentvariable}[1]{\texttt{\small #1}\index{#1@\texttt{#1}}}
\newcommand{\Indexfile}[1]{\mathit{indexname}.\texttt{#1}}
\newcommand{\Indexfilesixfr}[1]{\mathit{indexname.6fr}.\texttt{#1}}
\newcommand{\TIS}[0]{\mathit{tistab}}
\newcommand{\OIS}[0]{\mathit{oistab}}
\newcommand{\AboutVmatchcmd}[1]{}
\newcommand{\AboutMkdnasixcmd}[1]{#1}
\newcommand{\EXECUTE}[1]{}

\newenvironment{Output}{%
\begin{scriptsize}
\begin{alltt}}{%
\end{alltt}
\end{scriptsize}%
\addvspace{-\medskipamount}
}

\newenvironment{LargeOutput}{%
 \begin{footnotesize}
 \begin{alltt}}{%
 \end{alltt}
 \end{footnotesize}%
 \addvspace{-\medskipamount}
}

\lstset{language=C,
        basicstyle=\ttfamily\footnotesize,
        stringstyle=\ttfamily,
        keywordstyle=\ttfamily,
        commentstyle=\ttfamily\itshape,
        showstringspaces=false}

\author{Stefan Kurtz\thanks{\SKaffiliation}}

\title{\MM and \SEMM:\\
       \textbf{Finding Maximal Multiple Matches}\\
       \textbf{in Large Sequences}\\[2mm]
       \textbf{Manual}}
\begin{document}
\maketitle

\section{The Programs \MM and \SEMM}
The programs \MM and \SEMM both find \MMEM{s} in \(k\geq 2\) sequences 
\(G_{0},\ldots,G_{k-1}\). \MM additionally can compute multiple exact match 
sets. These are a more compact representation of \MMEM{s} avoiding a
combinatorial explosion in the output. \MM assumes that 
the sequences \(G_{0},\ldots,G_{k-1}\) have been preprocessed by one 
of the programs \MKV, \MKRC, or \MKDNASIX. \SEMM only works on indexes
preprocessed by \MKV. All three programs \MKV, \MKRC, or \MKDNASIX
deliver an index referred to by \emph{indexname} in the sequel.
The program \MKRC and \MKDNASIX are described in this manual.
For more information on \MKV and the underlying index technique, see the
\emph{Vmatch} website at \url{http://www.vmatch.de}.
Readers who are interested in the formal definitions of 
the notion used here, are referred to Appendix \ref{BasicNotions}.

\MM implements two different algorithms to compute \MMEM{s}:
The first algorithm is described in \cite{HOEH:KUR:OHL:2002}
and it runs in \(O(kn+r)\), where \(n\) is the total length of
all sequences \(G_{0},\ldots,G_{k-1}\) and \(r\) is the number
of right maximal multiple matches.
The second algorithm is described in \cite{KUR:LON:2004}
and it runs in \(O(\Size{\Sigma}kn+m)\), where \(\Sigma\) is the
sequence alphabet and \(m\) is the number of \MMEM{s}.
If the alphabet size is small (e.g.\ in case of DNA alphabets)
and the number of sequences is small, then \(m\) is usually much
smaller than \(r\), which implies that the second algorithm is faster
than the first. In all other cases we recommend to use option
\Showoption{algorithm} with argument \Showoptionarg{fast},
to trigger the first algorithm.
Additionally \MM implements an algorithm to compute multiple exact
match sets. This is triggered by option 
\Showoption{algorithm} with argument \Showoptionarg{memset}.
\MM and \SEMM provide an option \Showoption{rare} to compute
rare \MMEM{s}. \MM computes rare \MMEM{s} using an 
algorithm described in \cite{ABO:KUR:OHL:2006}. The corresponding algorithm
used in \SEMM is described in \cite{OHL:KUR:2007}. Note that in
\cite{ABO:KUR:OHL:2006}, the notion \textit{infrequent} is used as a synonym 
for rare.

\subsection{Options for \MM}

\begin{Showprogramwithoptions}{\MM}{}

\Option{units}{\Showoptionarg{unitfilename}}{
Specify \Showoptionarg{unitfilename} which describes sequence
units. In many cases, a genome consists of several sequence parts 
(e.g.\ contigs). When computing \MMEM{s}, it is often desired that these
sequence parts are considered as a single unit. To facilitate this, we
have developed a notation which allows to specify these units. 
To keep the notation simple, we suppose that all sequences which make up 
a unit, build a consecutive block in the index produced by the
program \MKV. As a consequence, their descriptions directly follows each other
in the sequence description file \Showoptionarg{indexname}.\texttt{des}, 
where \Showoptionarg{indexname} is the name of the index.
So we can specify the units by separating the appropriate blocks of
description lines in a copy of the file 
\Showoptionarg{indexname}.\texttt{des}.
To produce a \Showoptionarg{unitfilename} in the correct format,
one makes a copy of \Showoptionarg{indexname}.\texttt{des}
into a file, say \Showoptionarg{indexname}.\texttt{units},
and edits this copy using a standard text editor. Each unit
of sequences is separated from the next unit by a single line beginning
with the keyword \texttt{UNITSEP} (called unit separator line). For 
example, if the first five sequences make up the unit, then one inserts a 
unit separator line after the fifth description line. The first and the last
line of the unit description file are not allowed to be unit separator
lines. Moreover, please do not modify the descriptions themselves, since 
they are checked to be identical to the original description lines in 
\Showoptionarg{indexname}.\Keyword{des}.
}

\Option{range}{\Showoptionarg{i}~\Showoptionarg{l}~\mbox{$\lbrack\Showoptionarg{r}\rbrack$}}{
Specify that in the \Showoptionarg{i}th sequence of the index to search,
the match has to start between position \Showoptionarg{l} and 
position \Showoptionarg{r}. \Showoptionarg{i} must be a valid sequence number 
in the range $[0,k-1]$, where \(k\) is the number of sequences. 
\(l\) must be in the range \([0,q-1]\), where \(q\) is the length of
the sequence with number \(i\). If \Showoptionarg{r} is given, then
it must be in the range \([l,q-1]\).  
If the optional argument \Showoptionarg{r} is missing, then
\(r\) is interpreted as \(q-1\). This option can be used more than
once, even for the same sequence number.
}

\Option{l}{$\ell$}{
Specify the length parameter $\ell$. This must be a positive integer.
Only multiple matches of length at least $\ell$ are reported. This option is
mandatory.
}

\Option{d}{~~~}{
Report direct matches. This option can be combined with option \Showoption{p}.
}

\Option{p}{~~~}{
Report palindromic matches. In the first sequence \(G_{0}\),
a palindromic match always refers to the forward strand. On at least
one of the sequences \(G_{1},\ldots,G_{k-1}\), the corresponding position
refers to the reverse complemented strand, which is signified by
prepending the symbol \symbol{45} in front of the position. Computing
palindromic matches only works for two kinds of special indices:
\begin{itemize}
\item
a \emph{reverse complemented index} which was constructed for the given
DNA input sequences using an extra program \MKRC. 
\item
a \emph{six-frame translation index} which was constructed for the
DNA input sequences using an extra program \MKDNASIX.
\end{itemize}
These programs are described below. Option \Showoption{p} can be combined 
with option \Showoption{d}. If neither option \Showoption{d} nor 
\Showoption{p} is used, then only direct matches are reported.
}

\Option{count}{~~~}{
Only count the number of matches.
Do not show matches, but a distribution of their lengths.
}

\Option{algorithm}{\Showoptionarg{fast}$\mid$\Showoptionarg{memset}}{
If the argument \Showoptionarg{fast} is used, then the 
algorithm described in \cite{HOEH:KUR:OHL:2002} is used.
This computes matches faster in many cases, compared to the default 
algorithm, described in \cite{KUR:LON:2004}. 
If the argument \Showoptionarg{memset} is used, then multiple exact
match sets are computed. These represent a superset of all
\MMEM{s} in a more compact way.
}

\Option{unitol}{$\tau$}{
Specify the uniqueness tolerance \(\tau\) which is a non-negative
integer. The larger this value, the more tolerant the program is if multiple
matching substrings are not unique in the given input sequences. 
To understand the effects of this option,
consider the case that a particular substring, say \(u\), occurs 
\(q_{i}\geq 1\) times in
sequence \(G_{i}\) for \(i\in[0,k-1]\). Ignoring the effects of requiring
left maximality and right maximality, this results in 
\(\prod_{i=0}^{k-1}q_{i}\) multiple exact matches for \(u\). That is, 
there is a combinatorial explosion in the number of multiple matches, if 
\(u\) occurs more than once in any of the sequences \(G_{i}\). Note that 
\(\sum_{i=0}^{k-1}q_{i}\geq k\) because \(q_{i}\geq 1\).
The uniqueness tolerance specifies an upper bound \(k+\tau\) on this sum.
That is, when specifying \(\tau\) with this option, 
\(k+\tau\geq\sum_{i=0}^{k-1}q_{i}\) must hold.
Consider the case that \(\tau=0\). Then \(\sum_{i=0}^{k-1}q_{i}=k\) which
implies \(q_{i}=1\) for all \(i\in[0,k-1]\). As a consequence, 
only those \MMEM{s} are reported, whose matching substring 
occurs exactly once in each of the input sequences 
\(G_{0},G_{1},\ldots,G_{k-1}\). That is, it is a unique \MMEM. 
If \(\tau=1\), then in sequence \(G_{j}\) for some \(j\in[0,k-1]\), the 
multiple matching substring occurs twice, while in all 
\(G_{i}\), \(i\neq j\), it occurs exactly once.
If \(\tau=2\), then there are two alternatives:
\begin{itemize}
\item
in one sequence \(G_{j}\) for some \(j\in[0,k-1]\), 
the multiple matching substring occurs three times, while in all \(G_{i}\), 
\(i\neq j\), it occurs exactly once.
\item
in two different sequences \(G_{j}\) and \(G_{j'}\) for some 
\(j,j'\in[0,k-1]\), the multiple matching substring occurs twice,
while in all \(G_{i}\), \(i\neq j\) and \(i\neq j'\) it occurs exactly once.
\end{itemize}
For larger values of \(\tau\), corresponding argumentation holds.
}

\Option{rare}{\lbrack$i$\rbrack~$t_{i}$}{
Specify the rare occurrence threshold \(t_{i}\geq 1\) 
for genome \(i\in[0,k-1]\). That is, one specifies the rare occurrence 
threshold for a specific genome. If the first argument $i$ is missing, then
the given value $t_{i}$ applies to each \(i\in[0,k-1]\), except to
those \(i\in[0,k-1]\) for which the rareness occurrence was specified
otherwise. That is, the single argument use of this option
allows to specify the rare occurrence threshold for several genomes. 
This option can be used more than once. Only one 
single argument occurrence of this option is allowed.
To understand the effects of this option,
consider the case that a particular substring, say \(u\), occurs
\(q_{i}\geq 1\) times in sequence \(G_{i}\) for \(i\in[0,k-1]\). If this
option is specified then a \MMEM for \(u\) is only computed if 
\(q_{i}\leq t_{i}\). In other words, $t_{i}$ is a threshold on the number
of occurrences of multiple matching substring in \(G_{i}\).
}

\Option{selfun}{\Showoptionarg{filename}~\mbox{$\lbrack\Showoptionarg{extraargs}\rbrack$}}{
Access file \Showoptionarg{filename}, which must be a shared object file 
containing a multimatch selection function bundle. If the access to
such a function bundle is not possible, then the program 
terminates with
error code 1.  If \Showoptionarg{filename} is  not  an absolute path  
(i.e., it does not begin with a forward slash (`\texttt{\SY{47}}'), then
the file is searched for in the colon-separated list of directories specified
by the environment variable \LDlibpath.
For details on selection functions see Section \ref{Applications} and
Appendix \ref{SelectionConcept}.
\Showoptionarg{filename} must have the appropriate extension. This depends 
on the particular platform. For most Unix-platforms the extension
\Filenamesuffix{.so} is valid. For HP-UX, the valid file extension is 
\Filenamesuffix{.sl}. Option \Showoption{filename}
can be followed by an optional
list \Showoptionarg{extraargs} of extra arguments
(not beginning with the symbol \texttt{\symbol{45}}).
These are ignored by \VM and \VMS. Since they are stored in the
argument vector passed to the function \Selectionfunction{selectmatchHeader}
(see Section \ref{SelectionConcept}), they can be evaluated in the
selection function bundle.
}

\Option{maxnummatches}{$m$}{
Restrict the number of \MMEM{s} reported to at most \(m\). That is,
if there are more than \(m\) \MMEM{s} only the first \(m\) matches
in order of their computation are reported and the program stops.
}

\Option{showevalue}{~~~}{
Output the E-value of a \MMEM, according to the model described in
Appendix \ref{MMEMevalue}.
}

\Option{s}{$\lbrack$\Showoptionarg{q}$\rbrack$}{
Additionally report the matching substring.
The optional parameter value \Showoptionarg{q} specifies the linewidth. 
That is, the matching
substring is formatted to \Showoptionarg{q} symbols per line.
If \Showoptionarg{q} is not specified, then the default linewidth is 
60.
}

\Option{absolute}{}{
Show the absolute positions of the matches (i.e.\ with respect to 
the first position of the first sequence) instead of relative positions
(i.e.\ with respect to the first position of the corresponding sequence).
}

\Option{seqnumrel}{}{
If the option \Showoption{units} is used, then in the default case,
all positions are reported relative to the first position of the 
unit. If this option is used, then all positions are reported relative
to the first position of the sequence.
}

\Option{v}{~}{
Be verbose, that is, give reports about the different steps as well as the
resource requirements of the computation.
}

\Option{version}{}{
Show the version of the program, and report the compilation date as well
as the compilation options.
}

\Option{help}{}{
Show a summary of all options and terminate.
}
\end{Showprogramwithoptions}

Now note the following rules when combining options:
\begin{enumerate}
\item
Each of the options can be specified only once, except for option
\Showoption{range} which can be used multiple times.
\item
If option \Showoption{count} is used, then
any of the options  
\Showoption{unitol},
\Showoption{selfun},
\Showoption{maxnummatches},
\Showoption{showevalue},
\Showoption{s},
\Showoption{absolute},
\Showoption{seqnumrel}, and
\Showoption{p}
\Showoption{rare}
cannot be used.
\item
Option \Showoption{seqnumrel} is incompatible with option
\Showoption{absolute}.
\item
Option \Showoption{range} is used, then option 
\Showoption{p} and \Showoption{unitol} cannot be used.
\item
Option \Showoption{seqnumrel} requires that option
\Showoption{units} is used.
\item
Option \Showoption{rare} and \Showoption{unitol} cannot be 
combined with option \Showoption{algorithm memset}.
\end{enumerate}

\subsection{Options for \SEMM}

The program is called as follows:
\par
\noindent\SEMM [\emph{options}] \emph{idx} 
\texttt{[+-]}$\mathit{f}_{1}$
\texttt{[+-]}$\mathit{f}_{2}$\ldots
\texttt{[+-]}$\mathit{f}_{k}$
\par
where \emph{idx} is the name of an index computed by \MKV and 
\(\mathit{f}_{1},\mathit{f}_{2},\ldots,\mathit{f}_{k}\)  are 
\(k\geq 1\) query files. Note that the name of each query file can optionally
be prefixed by the symbol \texttt{+} or \texttt{-} to specify that the
file is matched on the forward strand (symbol \texttt{+}) or the reverse
strand (symbol \texttt{-}). If neither  \texttt{+} nor \texttt{-} is used,
then the corresponding query file is matched on the forward strand. Note
that no white space is allowed between the sign-symbols and the filename.
Also note that the symbols cannot be combined. If it is required to
compute rare \MMEM{s} for all $2^{k}$ possible combinations of strands,
then use the the perl-script \MEMSPEall, described below in Section
\ref{MEMSPEallsection}.

Suppose that one wants to compute rare multiple
exact matches between \(k+1\) sequences. Then one has to compute an 
index, say \texttt{idx}, for the first sequence file \(f_{0}\). This
can be done using the following call to \MKV:

\begin{small}
\texttt{mkvtree -indexname idx -dna -tis -sti1 -bck -suf -lcp -pl -v -db }$f_{0}$
\end{small}

Given the index one calls \SEMM as specified above. \SEMM allows the following
options: 
\par
\begin{Justshowoptions}
\item
\Option{l}{$\ell$}{
Specify the length parameter $\ell$. See same option for \MM for more
details.
}

\Option{phase}{(\Showoptionkey{search}$\mid$\Showoptionkey{output})}{
specify that either only matches are searched (option argument
\Showoptionkey{search}), or that matches are output (option argument
\Showoptionkey{output}). If it is specified that matches are
only searched, then for all query sequences \(f_{1},f_{2},\ldots,f_{k}\),
the result of the search is stored in a file named 
\texttt{idx.}$f_{i}\mathit{flag}$match, where \(\mathit{flag}\) is \texttt{+},
if \(f_{i}\) was matched on the forward strand, and \texttt{+}, if it was
matched on the reverse strand.
}

\Option{rare}{$i$~$t_{i}$}{
Specify the rare occurrence threshold \(t_{i}\geq 1\) 
for genome \(i\in[0,k-1]\).  See same option for \MM for more details.
}

\Option{showevalue}{~~~}{
Output the E-value of a \MMEM, according to the model described in
Appendix \ref{MMEMevalue}.
}

\Option{s}{$\lbrack$\Showoptionarg{q}$\rbrack$}{
Additionally report the matching substring.  See same option for \MM for more
details.
}

\Option{v}{~}{
Be verbose, that is, give reports about the different steps as well as the
resource requirements of the computation.
}

\Option{version}{}{
Show the version of the program, and report the compilation date as well
as the compilation options.
}

\Option{help}{}{
Show a summary of all options and terminate.
}
\end{Justshowoptions}
The following options available for \MM may also be useful for
\SEMM: \Showoption{range}, 
\Showoption{selfun},
\Showoption{count},
\Showoption{maxnummatches}. If you would like to use these options
for \SEMM, please contact me.

\section{The perl-script \MEMSPEall}\label{MEMSPEallsection}
The perl-script \MEMSPEall iterates \SEMM for a given index and a 
list of query files to compute all matches for all combinations
of strands. The first argument for \MEMSPEall is 
the path to \SEMM, the second is a string serving as the prefix
for the resultfiles produced by the different calls to 
\SEMM. That is, if this argument is the string \texttt{result},
then the output for the different calls is stored in
files which begin with \texttt{result} and ending with a 
sequence of length \(k\) over the the symbol set \(\{\texttt{+},\texttt{-}\}\).
For example, the matches for the run with 3 query files all matched on the
forward strand is stored in \texttt{result+++}. The matches for the run with 3
query files where the first is matched on the reverse strand, while
the other are matched on the forward strand is stored in file
\texttt{result-++}. The third argument for \MEMSPEall is the minimum length
of the matches to be searched. The fourth parameter is specify the 
rare parameter, which is common to all sequences. The remaining arguments
specify the index and the query files.

\section{The Program \MKRC}
\MKRC is required whenever you want to compute reverse complemented
multiple matches. \MKRC generates from the given input
files an index (i.e.\ a collection of files) which allow \MM to compute
reverse complemented \MMEM{s}. In these, in at least one (but not all) 
sequences, the matching substring is on the reverse strand. 

\begin{Showprogramwithoptionswithoutindex}{\MKRC}{}

\Option{db}{\Showoptionarg{dbfiles}}{
Specify a non empty list of database files separated by white spaces.
Each database file contains sequences in one of the following formats:
Fasta, Genbank, EMBL, and SWISSPROT. The user does not have to 
specify the input format. However, the format of all files has to 
be identical. The sequence must consist of characters over the alphabet
\texttt{a}, \texttt{c}, \texttt{g}, \texttt{t}, or \texttt{u} (in
lower or upper case), or wildcards
\texttt{n}, \texttt{s}, \texttt{y}, \texttt{w}, \texttt{r}, \texttt{k},
\texttt{v}, \texttt{b}, \texttt{d}, \texttt{h}, \texttt{m}. 
White spaces in the input files are ignored. This option is mandatory.
This option is identical with the same option 
of the program \MKV.
}

\Option{indexname}{\Showoptionarg{filepath}}{
Specify \Showoptionarg{filepath} to be the name of the index, later referred
to by \emph{indexname}. This option is mandatory, if more than one database 
file is given.
If there is only one database file, and this option is not given, then 
\emph{indexname} is the basename of the given \Showoptionarg{filepath},
i.e.\ the filename stripped by the directory path where it is stored. 
\Showoptionarg{filepath} can be a complete path.
}

\Option{v}{~~~}{%
Be verbose, that is, give reports about the different steps as well as the
resource requirements of the computation. This option is recommended.
}

\Option{version}{~~~}{
Show the version of the program.
Also report the compilation date and the compilation options.
}

\Option{help}{~~~}{
Show a summary of all options and terminate.
}
\end{Showprogramwithoptionswithoutindex}


\section{The Program \MKDNASIX}\label{MKDNASIXSECTION}
\MKDNASIX is very similar to \MKV. While \MKV can handle sequences
over arbitrary alphabets, \MKDNASIX requires DNA-sequences as input.
It generates two indices, namely:
\begin{itemize}
\item
A flat index \emph{indexname} for the the given DNA sequences.
It mainly consists of the two files 
\(\Indexfile{tis}\) and \(\Indexfile{ois}\). This index
is mainly used for output purpose.
\item
An index \emph{indexname.6fr} for the given DNA sequences
translated in all six reading frames. This is used for
computing the matches.
\end{itemize}
These two indices allow \VM to compute matches on the protein level.

\begin{Showprogramwithoptionswithoutindex}{\MKDNASIX}{}

\Option{db}{\Showoptionarg{dbfiles}}{
Specify a non empty list of database files separated by white spaces.
Each database file contains sequences in one of the following formats:
Fasta, Genbank, EMBL, and SWISSPROT. The user does not have to 
specify the input format. However, the format of all files has to 
be identical. The sequence must consist of characters over the alphabet
\texttt{a}, \texttt{c}, \texttt{g}, \texttt{t}, or \texttt{u} (in
lower or upper case), or wildcards
\texttt{n}, \texttt{s}, \texttt{y}, \texttt{w}, \texttt{r}, \texttt{k},
\texttt{v}, \texttt{b}, \texttt{d}, \texttt{h}, \texttt{m}. 
White spaces in the input files are ignored. This option is mandatory.
This option is identical with the same option 
of the program \MKV.
}

\Option{smap}{\Showoptionarg{mapfile}}{
Specify the file storing the symbol mapping applied
to the amino acid symbols after the six frame translation.
\index{symbol mapping}
If the given \Showoptionarg{mapfile} cannot be found in the directory where
\MKV is run, then all directories specified by the environment variable
\Environmentvariable{MKVTREESMAPDIR}
are searched. If defined correctly, this contains a
list of directory paths separated by colons (\texttt{`:'}). For example, 
if one uses the \Programname{csh} or the \Programname{tcsh}, 
the definition of the environment-variable could look like this:

\begin{LargeOutput}
setenv MKVTREESMAPDIR \SY{34}\$HOME/vstree:/usr/vstree/TRANS\SY{34}
\end{LargeOutput}

For the \Programname{bash} or the \Programname{sh} the 
definition could look like

\begin{LargeOutput}
MKVTREESMAPDIR=\SY{34}\$HOME/vstree:/usr/vstree/TRANS\SY{34}

export MKVTREESMAPDIR
\end{LargeOutput}

Then, if \Showoptionarg{mapfile} is not available in the current directory, 
\MKV searches for \Showoptionarg{mapfile} in the two given directories. It
scans the directory-list from left to right. As soon as it has found the file
it stops. If the file cannot be found, an error message is reported and
the program exits with \index{error code} error code 1. 
\AboutMkdnasixcmd{See the manual of the program Vmatch
(\url{http://www.vmatch.de})}
\AboutVmatchcmd{See Appendix \ref{Symbolmap}}
for a more detailed explanation of the format of the symbol mapping file.
}

\Option{transnum}{\Showoptionarg{$t$}}{
Specify the number of the codon translation table used for
the six frame translation. $t$ must be a number in the range \([1,23]\) except
for 7, 8, 17, 18, 19 and 20. Table \ref{Codontranstables} gives the possible numbers
and their names. The codon translation tables, their numbers, and their
names were taken from \url{ftp://ftp.ncbi.nih.gov/entrez/misc/data/gc.prt}.
If this option is not used, then the standard codon translation
table (number 1) is used.
}

\Option{indexname}{\Showoptionarg{filepath}}{
Specify \Showoptionarg{filepath} to be the name of the index, later referred
to by \emph{indexname}. This option is mandatory, if more than one database 
file is given and if additionally at least one file comprising the 
index is stored (i.e.\ if any of the output options is used). If no 
file from the index is stored, then this option is not allowed. 
If there is only one database file, and this option is not given, then 
\emph{indexname} is the basename of the given \Showoptionarg{filepath},
i.e.\ the filename stripped by the directory path where it is stored. 
The \Showoptionarg{filepath} can be a complete path.
}

\Option{tis}{}{
Store the table \(\TIS\) in file \(\Indexfilesixfr{tis}\).
}

\Option{ois}{}{
Store the table \(\OIS\) in file \(\Indexfilesixfr{ois}\).
}

\Option{v}{~~~}{%
Be verbose, that is, give reports about the different steps as well as the
resource requirements of the computation. This option is recommended.
}

\Option{version}{~~~}{
Show the version of the program.
Also report the compilation date and the compilation options.
}

\Option{help}{~~~}{
Show a summary of all options and terminate.
}

\end{Showprogramwithoptionswithoutindex}

\input{codontrans}


\section{Example Applications}\label{Applications}
Suppose file \Filename{seqs4} contains the following sequences in fasta-format:

\begin{LargeOutput}
>seq0
ctttgactgtagttt
>seq1
cggtttaca
>seq2
cttttgt
>seq3
ggtttagt
\end{LargeOutput}

We use the program \MKV to generate the index required for \MM:

\EXECUTE{mkvtree -pl -suf -bwt -lcp -tis -dna -db seqs4 -v}

\subsection{Simple Usage of \MM}

The simplest call of \MM just uses the option \Showoption{l}:

\EXECUTE{multimat -l 3 seqs4}

Here we have \(k=4\). Each maximal multiple match 
\((l,p_{0},p_{1},p_{2},p_{3})\) is shown on a single line
as \(l~p_{0}~p_{1}~p_{2}~p_{3}\). Additionally using the option
\Showoption{s} in the call of \MM, also the matching substrings are reported.

\EXECUTE{multimat -l 3 -s seqs4}

Sometimes it is helpful to report E-values, using the option
\Showoption{showevalue}.

\EXECUTE{multimat -l 3 -showevalue seqs4}

Alternatively, we can count the distribution of the match lengths, using the
option \Showoption{count}. We use 1 for the minimum match length
(not recommended in real world applications).

\EXECUTE{multimat -l 1 -count seqs4}

This means that there are 
398 \MMEM{s} of length 1 (89.24\% of all \MMEM{s} found),
44 \MMEM{s} of length 2 (9.87\% of all \MMEM{s} found), and
4 \MMEM{s} of length 3 (0.9\% of all \MMEM{s} found). Also the 
cumulative value of the distribution is reported. For example,
99.10\% of all \MMEM{s} reported are  of length at most 2.

Now suppose we want to restrict the positions where to find a
match. Let us restrict the range of positions from 1 to 5 in the
first sequence (which has number 0):

\EXECUTE{multimat -l 3 -range 0 1 5 seqs4}

Note that the two \MMEM{s} starting at position 12 in the first
sequence are not reported, because of the specified range restriction.

Now consider a larger sequence file \Filename{ATsmall} containing 11 input
sequences. Calling \MKV, we compute the appropriate index:

\EXECUTE{mkvtree -pl -suf -bwt -lcp -tis -dna -v -db ATsmall}

The longest \MMEM{s} are of length 5:

\EXECUTE{multimat -l 5 ATsmall}

One easily recognizes that all multimatches are for the
same substring, and that the positions only vary in
sequence 4, 5, and 8 (counting from 0). Using the option
\Showoption{algorithm}~\Showoptionarg{memset} allows to
reduce the above list of matches to a single multiple exact match set:

\EXECUTE{multimat -s -l 5 -algorithm memset ATsmall}

Note that for sequences 4, 5, and 8 the position sets \(P_{4}\),
\(P_{5}\) and \(P_{8}\) corresponding to sequences \(G_{4}\),
\(G_{5}\) and \(G_{8}\) are shown in standard mathematical notation
within a pair of curly brackets and comma separated elements. 
The other position sets only contain one position, so that the pair of
curly brackets is omitted.

\subsection{Simple Usage of \SEMM}

To show the use of \SEMM, we split the file \texttt{ATsmall} into 11 files 
and only use the first four of these files named
\texttt{\small ATs-00}, 
\texttt{\small ATs-01},
\texttt{\small ATs-02}, and
\texttt{\small ATs-03}.

We only index the first file as follows:

\EXECUTE{mkvtree -indexname ATS -dna -tis -sti1 -bck -suf -lcp -pl -v -db ATs-00}

and use the other three files as queries. This time we look for 
matches of length at least 6 such that the match occurs at most
5 times in each of the 4 sequences:

\EXECUTE{ramaco -l 6 -rare 5 ATS ATs-01 ATs-02 ATs-03}

That is, we have found 3 rare \MMEM{s}. The 5 numbers show the length
and the positions where the matches occur in the corresponding sequence. 
The first number is the length, the second position
refers to the sequence in the index, the third refers to the
first query, etc.

We split the search and the output phase by calling 
\SEMM two times. The only difference in the calls is 
the argument top option \Showoption{phase}.

\EXECUTE{ramaco -v -phase search -l 6 -rare 5 ATS ATs-01 ATs-02 ATs-03}

This generates three files \texttt{\small ATS.ATs-01+match}, 
\texttt{\small ATS.ATs-02+match}, and \texttt{\small ATS.ATs-03+match}. 
These are used in the second call to produce the output:

\EXECUTE{ramaco -v -phase output -l 6 -rare 5 ATS ATs-01 ATs-02 ATs-03}

Note that the three files intermediate files \texttt{\small ATS.ATs-01+match}, 
\texttt{\small ATS.ATs-02+match}, and \texttt{\small ATS.ATs-03+match} can also 
be generated by three separate calls to \SEMM: 

\begin{Output}
ramaco -phase search -l 6 -rare 5 ATS ATs-01
ramaco -phase search -l 6 -rare 5 ATS ATs-02
ramaco -phase search -l 6 -rare 5 ATS ATs-03
\end{Output}

These three search-calls can be done in parallel, since they do not
depend on each other.

If additionally one want to compute matches on the reverse strand of the
sequences in \texttt{\small ATs-01}, then one calls

\EXECUTE{ramaco -v -phase search -l 6 -rare 5 ATS -ATs-01}

to obtain the output as follows:

\EXECUTE{ramaco -phase output -l 6 -rare 5 ATS -ATs-01 ATs-02 ATs-03}

Note that for this call the intermediate files 
\texttt{\small ATS.ATs-02+match} and \texttt{\small ATS.ATs-03+match} 
are reused. More general, for $k$ queries we have to call
\texttt{ramaco -phase search ...} \(2\cdot k\) times, and
\texttt{ramaco -phase output ...} \(2^{k}\) times to produce the matches
for all possible strand combinations. This is what the script
\MEMSPEall does.

\subsection{\MM with units, palindromic and 6-frame indexes}

To show the use of the option \textit{units}, we consider the
file \Filename{ATsmall.des}, produced by the last but one call to
\MKV. The file contains the sequence description and looks as follows:

\EXECUTE{cat ATsmall.des}

Note that we have cut each line after 70 characters. We make a copy of
this file

\begin{LargeOutput}
cp ATsmall.des ATsmall.units
\end{LargeOutput}

and edit \Filename{ATsmall.units}, inserting the keyword \Keyword{UNITSEP}
after every second sequence. We obtain the following file contents:

\EXECUTE{cat ATsmall.units}

Thus the sequences (identified by their \emph{gi}-number)
are grouped into 6 units as follows:
\begin{center}
\begin{tabular}{l}
\(\{\Keyword{5587835},\Keyword{4714049}\}\),\\
\(\{\Keyword{4714048},\Keyword{4714047}\}\),\\
\(\{\Keyword{4714046},\Keyword{4714045}\}\),\\
\(\{\Keyword{4714044},\Keyword{4714043}\}\),\\
\(\{\Keyword{4714042},\Keyword{4714041}\}\),\\
\(\{\Keyword{4714040}\}\).
\end{tabular}
\end{center}

Now option \Showoption{units} allows to compute \MMEM{s} between the 6 units.
Each position is reported relative to the first position of the first
sequence in the unit. For example, position 1002 in the first \MMEM is 
relative to the start of the sequence with \emph{gi}-number \Keyword{4714048}.

\EXECUTE{multimat -s -l 7 -units ATsmall.units ATsmall}

There are many more \MMEM{s} if we reduce the length threshold:

\EXECUTE{multimat -count -l 5 -units ATsmall.units ATsmall}

However, if we restrict to unique \MMEM{s}, then only one match remains:

\EXECUTE{multimat -unitol 0 -s -l 5 -units ATsmall.units ATsmall}

We obtain the same result using the rare option for the first three units:

\EXECUTE{multimat -rare 1 -s -l 5 -units ATsmall.units ATsmall}

If we relax the rare-parameter for the first unit, we obtain 9 matches:

\EXECUTE{multimat -rare 0 2 -rare 1 -l 5 -units ATsmall.units ATsmall}

Next we want to compute palindromic \MMEM{s}. To do so, we have to compute
a reverse complemented index using the program \MKRC. We do this for 
the small sequence set of the initial example:

\EXECUTE{mkrcidx -db seqs4 -v}

We use the option \Showoption{p} for \MM:

\EXECUTE{multimat -l 2 -p -s seqs4.rcm}

Consider the first \MMEM. It is of length 2. In the first sequence
contained in file \Filename{seqs4}, namely \texttt{ctttgactgtagttt},
the substring \texttt{ac} occurs at position 5 (counting from 0). The
reverse complement \texttt{gt} of \texttt{ac} occurs at position 2 in
the second sequence \texttt{cggtttaca}, at position 5 in the third
sequence \texttt{cttttgt}, and at position 6 in the fourth sequence
\texttt{ggtttagt}. Hence the positions 2, 5, and 6 are shown as 
negative numbers.

Next we want to compute multiple matches in a set of DNA sequences
which occur on the protein level. To do so, we have to compute
a six-frame translation index using the program \MKDNASIX. We do this for 
a file \texttt{seqs3} with 3 sequences.
Since we want to relax the matching on the protein level, we specify a
symbol map \texttt{TransProt11}, when constructing the six-frame translation
index:

\EXECUTE{mkdna6idx -tis -ois -db seqs3 -smap TransProt11 -v}

We use the options \Showoption{d} and \Showoption{p} for \MM, applied to
the six-frame index \texttt{seqs3.6fr}.

\EXECUTE{multimat -l 15 -d -s seqs3.6fr}

Thus there is a \MMEM of length 15. Since the match is on the translated
codons, the DNA-substrings are not necessarily the same. Hence for each 
sequence they are reported. Furthermore, the last line of the 
match reports the corresponding amino acid sequence corresponding
to the first dna sequence.

The concept of selection functions is explained in Appendix 
\ref{SelectionConcept}. Here we give an example.

Consider the following functions defined in a file 
\Filename{selmulticontext.c}:

\begin{lstlisting}{}
Sint selectmultimatchHeader(int argc,const char **argv)
{
  int i;

  printf("#");
  for(i=0; i<argc; i++)
  {
    printf(" %s",argv[i]);
  }
  printf("\n");
  return 1;   // map the transformed input sequence
}

Sint selectmultimatch(Virtualtree *virtualtree,
                      Uint *unitboundaries,
                      Uint numofunits,
                      Multimatch *multimatch)
{
  if(virtualtree->multiseq.sequence == NULL)
  {
    fprintf(stderr,"selectmultimatch: sequence not read: use option -s\n");
    return (Sint) -1;
  }
  // does matching sequence begin with first character of alphabet?
  if(virtualtree->multiseq.sequence[multimatch->positions[0]] == 0)
  {
    return 1;
  }
  return 0;
}
\end{lstlisting}

The function \Selectionfunction{selectmultimatchHeader} is called before
the index is read. The arguments \lstinline{argc} and \lstinline{argv}
allow access to the arguments \MM was called with. The function first
checks if the input sequence is available in main memory. If not,
then an error is produced. Then it simply echoes all given
arguments to stdout. The function
The function \Selectionfunction{selectmultimatch} is applied to each match
match found. If matching sequence begins with the first character of
the alphabet (s indicated by comparison with 0), then the function
returns \lstinline{1}, which means that \MM echoes the match to 
stdout. If the \Selectionfunction{selectmultimatch} is applied to each
\MMEM found. If the matching sequence does not begin with the first character of
the alphabet, then  the function returns \lstinline{0}, which tells
\MM to not output the \MMEM{}. 

To make use of these functions we have to compile a shared object.
This is done by appropriately calling a
compiler, as specified in \lstinline{SELECT/makefile}. For example,
on Linux we use \texttt{gcc}-compiler with the following options:

\begin{LargeOutput}
gcc -Wall -Werror -O3 -shared selmulticontext.c -o selmulticontext.so
\end{LargeOutput}

This produces the shared object \Filename{selmulticontext.so}, which we
use as an argument to option \Showoption{selfun} when calling \MM:

\EXECUTE{multimat -s -l 6 -selfun selmulticontext.so -units ATsmall.units ATsmall | head -n 12}

As expected, the arguments of \MM are echoed, and all \MMEM{s} are
output, whose matching sequence starts with the first character of 
the alphabet.

Other examples for selection functions can be found in the subdirectory
\Filename{SELECT}. This also contains a \Filename{makefile}, showing how to
compile a shared object for the supported platforms.

\appendix

\section{Basic Notions}\label{BasicNotions}
We index the characters of a sequence from \(0\). That is, a sequence
\(S\) of length \(n\) is written as 
\(S=\Subchar{S}{0}\Subchar{S}{1}\ldots\Subchar{S}{n-1}\).
A \emph{substring} of \(S\) is a sequence \(\Subchar{S}{i}\ldots\Subchar{S}{j}\)
for some \(i,j\in[0,n-1]\). If \(i>j\), then
\(\Subchar{S}{i}\ldots\Subchar{S}{j}\) denotes the empty
string. \(\Subchar{S}{i}\ldots\Subchar{S}{j}\) is also written as
\(\Substring{S}{i}{j}\).
A \emph{prefix} of \(S\) is a sequence \(\Substring{S}{0}{i}\) for some
\(i\in[0,n-1]\). A \emph{suffix} of \(S\) is a sequence 
\(\Substring{S}{j}{n-1}\) for some \(j\in[0,n-1]\).
Consider a set \(\{G_{0},\ldots,G_{k-1}\}\) of \(k\geq 2\) sequences,
the genomes. A \emph{multiple exact match}
is a \((k+1)\)-tuple \(\MultiMatch{l}{p}\) such that \(l>0\),
\(p_{q}\in[0,\Size{G_{q}}-l]\), and
\(\Substring{G_{q}}{p_{q}}{p_{q}+l-1}=\Substring{G_{q'}}{p_{q'}}{p_{q'}+l-1}\)
for all \(q,q'\in[0,k-1]\). If \(\MultiMatch{l}{p}\) is a multiple exact match,
then \(\Substring{G_{q}}{p_{0}}{p_{0}+l-1}\) is the \emph{matching substring}.
A multiple exact match is \emph{left maximal} if
for at least one pair \((q,q')\in[0,k-1]\times[0,k-1]\),
we either have \(p_{q}=0\), or \(p_{q'}=0\), or
\(\Subchar{G_{q}}{p_{q}-1}\neq\Subchar{G_{q'}}{p_{q'}-1}\).
A multiple exact match is \emph{right maximal} if
for at least one pair \((q,q')\in[0,k-1]\times[0,k-1]\),
we either have
\(p_{q}+l=\Size{G_{q}}\), or
\(p_{q'}+l=\Size{G_{q'}}\), or
\(\Subchar{G_{q}}{p_{q}+l}\neq\Subchar{G_{q'}}{p_{q'}+l}\).
A multiple exact match is \emph{maximal} if it is left maximal and right
maximal. A maximal multiple exact match is also called \MMEM.
For \(k=2\), we often use the notion \MEM.
Roughly speaking, a \MMEM is a sequence of length $l$ that
occurs in all sequences $G_{0},\ldots,G_{k-1}$
(at positions $p_{0},\ldots,p_{k-1}$),
and cannot simultaneously be extended to the left or to the right
in every sequence.

Let \(w\) be a sequence of length \(l\) occurring as a substring 
in all sequences
\(G_{0},\ldots,G_{k-1}\). For each \(i\in[0,k-1]\), let
\(P_{i}\) be the set of positions in $G_{i}$ where \(w\) occurs.
Note that by assumption \(P_{i}\neq\emptyset\) for \(i\in[0,k-1]\).
The \((k+1)\)-tuple \((l,P_{0},\ldots,P_{k-1})\) is called
\emph{multiple exact matching set} for \(w\), abbreviated
\MEMset. Note that each \MMEM is an element of 
\(P_{0}\times\cdots\times P_{k-1}\).
However, due to the maximality constraint, there may be elements
in \(P_{0}\times\cdots\times P_{k-1}\) which do not correspond to a \MMEM.

\section{Significance of \MMEM{s}}\label{MMEMevalue}
In order to assess the significance of a \MMEM,
we compute its E-value, i.e., the number of \MMEM{s} of
the same length or longer, that one would expect to find in 
\(k\) random sequences of the same length.
As a model of random sequences, we assume a uniform Bernoulli
model, where each base has the same probability of occurrence.

Let \(\sigma=\Size{\Sigma}\). Assume a fixed
choice of positions \(p_{0},\ldots,p_{k}\), where
\(p_{q}\in[0,\Size{G_{q}}-1]\) for \(q\in[0,k-1]\). We now 
consider the following events:
\begin{itemize}
\item
the event of a single character match at these positions, i.e.\
\(\Subchar{G}{p_{q}}=\Subchar{G}{p_{q'}}\) for each pair 
\((q,q')\in[0,k-1]\times[0,k-1]\).
\item
the event of a single character mismatch at these positions, i.e.\
\(\Subchar{G}{p_{q}}\neq\Subchar{G}{p_{q'}}\) for at least one pair 
\((q,q')\in[0,k-1]\times[0,k-1]\).
\end{itemize}
Obviously, there are \(\sigma\) events of single character matches and 
\(\sigma^{k}-\sigma\) events of single character mismatches.
Thus, the probability of a single character match is 
\[
\frac{\sigma}{(\sigma^{k}-\sigma)+\sigma}
    =\frac{\sigma}{\sigma^{k}}
    =\frac{1}{\sigma^{k-1}}
\]
Hence, the probability of a single character mismatch is 
\(1-\frac{1}{\sigma^{k-1}}\). Let \(p=\frac{1}{\sigma^{k-1}}\). 

It is easy to see that the probability of a
multiple exact match of length exactly \(\ell\)
at positions \(p_{0},\ldots,p_{k}\), 
\(p_{q}\in[0,\Size{G_{q}}-\ell]\), \(q\in[0,k-1]\),
is \(p^{\ell}\).

Based on these results, it is not difficult to calculate the
sought probability. At first, observe that the number of \MMEM{s} of 
length $\geq \ell $ is the same as the number of (only) left-maximal 
multiple exact match of length exactly $\ell$. Moreover, ignoring
boundary effects, for each \(q\in[0,k-1]\), there are 
\(n_{q}:=\Size{G_{q}}-(\ell-1)\) choices of \(p_{q}\) in \(G_{q}\). 
Hence, we obtain
\begin{eqnarray}
\lefteqn{\Expect[\# \mbox{ of \MMEM{s} of length $\geq\ell$}]}\nonumber \\
& = &
\Expect[\# \mbox{ of left-maximal multiple exact matches of length $\ell$}]\nonumber \\
& = &
\sum_{q\in[0,k-1],p_{q}\in[0,\Size{G_{q}}-\ell+1]}
  \PR [
  {\setlength{\arraycolsep}{0em}
   \begin{array}[t]{l}
     \mbox{multiple exact match at positions }p_{0},\ldots,p_{k-1}\mbox{ and}\\
     \mbox{mismatch at positions }p_{0}+\ell,\ldots,p_{k-1}+\ell]
   \end{array}
  }\nonumber \\
& = &
\sum_{q\in[0,k-1],p_{q}\in[0,\Size{G_{q}}-\ell+1]}
 \SingleCM^{\ell}(1-\SingleCM)\nonumber\\
& = &
\prod_{q=0}^{k-1}n_{q}~~~ \SingleCM^{\ell} (1-\SingleCM).\label{MultiProb}
\end{eqnarray}

\subsection{Computation of E-values}
The dominating expressions in (\ref{MultiProb}) are
\(\prod_{q=0}^{k-1}n_{q}\) and \(p^{\ell}\). The former can be a
very large value while the latter can be a very small value. To prevent 
overflows in the computation, we determine
\(l_{q}=\log_{\sigma}n_{q}\) for each \(q\in[0,k-1]\).
Then we evaluate \(m=\sum_{q\in[0,k-1]}l_{q}-(k-1)\ell\) and we obtain
\begin{eqnarray*}
\sigma^{m}&=&\sigma^{\sum_{q\in[0,k-1]}l_{q}-(k-1)\ell}\\
          &=&\sigma^{l_{0}}\cdot \ldots \cdot \sigma^{l_{k-1}} \cdot
                                              \frac{1}{\sigma^{(k-1)\ell}}\\
          &=&\prod_{q=0}^{k-1}n_{q}~~~\SingleCM^{\ell}.
\end{eqnarray*}
Thus \(\sigma^{m}(1-\SingleCM)\) is the E-value to be computed. 
The E-values are precomputed in a table \(\MMevalues\) indexed by
\(l\). This table is defined by:
\[
\MMevalues[l]=(1-p)\cdot\sigma^{\sum_{q\in[0,k-1]}l_{q}-(k-1)l}\\
\]
Obviously, the following equations hold:
\begin{eqnarray*}
\MMevalues[0]&=&(1-p)\cdot \sigma^{\sum_{q\in[0,k-1]}l_{q}}\\
\MMevalues[l+1]
&=&(1-p)\cdot \sigma^{\sum_{q\in[0,k-1]}l_{q}-(k-1)(l+1)}\\
&=&(1-p)\cdot \sigma^{\sum_{q\in[0,k-1]}l_{q}-((k-1)l+(k-1))}\\
&=&(1-p)\cdot \sigma^{\sum_{q\in[0,k-1]}l_{q}-(k-1)l-(k-1)}\\
&=&(1-p)\cdot \sigma^{\sum_{q\in[0,k-1]}l_{q}-(k-1)l}\cdot\sigma^{-(k-1)}\\
&=&\MMevalues[l]\cdot\frac{1}{\sigma^{k-1}}\\
&=&\MMevalues[l]\cdot p\\
\end{eqnarray*}
The first entry in \(\MMevalues\) can be computed in \(O(k)\) steps.
All other entries can be computed from the previous value in constant
time. The entries in \(\MMevalues\) become smaller with growing values
of \(l\). We compute them until we reach the E-value 0.0. Assuming that
we compute more than \(k\) values in \(\MMevalues\), the computation time
is proportional to the size of the table.

\section{Selection Function Bundles}\label{SelectionConcept}
To provide flexible and fast on-the-fly postprocessing 
of the \MMEM{s} without extra effort for output and parsing, \MM provides
the concept of \emph{selection functions}. This is very similar to 
the corresponding concept in \VM (see the \VM-manuals for justification 
of the concept). Selection functions allow to access the matches and 
the sequences involved in the match immediately after they are computed. The
selection is applied before further output is done.

Technically, a selection
function is specified by a bundle of functions written in the programming
language C. The program code implementing the selection functions
is compiled independently from \MM, but based on the data structures 
used in \MM. Using the option \Showoption{selfun}, \MM calls the 
selection functions via shared library calls (sometimes also called dynamic 
libraries). Since the definition of a selection function bundle requires a 
basic knowledge in C-programming, this concept is probably not suited
for the average user. Nevertheless, we describe this concept 
carefully to give an idea of the flexibility it allows.
\index{shared object}

A \emph{selection function bundle} consists of four functions
with the following function headers:
\index{selection function!bundle}
\begin{LargeOutput}
Sint selectmultimatchHeader(int argc,const char **argv);

Sint selectmultimatchInit(Virtualtree *virtualtree,
                          Uint *unitboundaries,
                          Uint numofunits);

Sint selectmultimatch(Virtualtree *virtualtree,
                      Uint *unitboundaries,
                      Uint numofunits,
                      Multimatch *multimatch);

Sint selectmultimatchWrap(Virtualtree *virtualtree,
                          Uint *unitboundaries,
                          Uint numofunits);

\end{LargeOutput}

\Datatype{Sint} is a type synonym for signed integers, and
\Datatype{Uint} is a type synonym for unsigned integers. Both are defined
the header file \Filename{types.h}.
The two C-structures \Datatype{Virtualtree} and \Datatype{Multimatch}
are defined in different header files. These are part of the distribution 
of the programs described in this manual.
If the option \Showoption{units} is specified when calling the program
\MM, then the parameter \Datatype{unitboundaries} refers to an array 
storing the boundary positions for \Datatype{numofunits} units.

Now suppose that these functions are specified in an extra
file, and that they are compiled in the appropriate way (see below). 
Then the following holds:
\begin{itemize}
\item
The function \Selectionfunction{selectmultimatchHeader} is called \emph{before} 
the index is read. The first argument of the 
function is the number of arguments of the corresponding call of 
\MM and the second argument is the corresponding argument 
vector storing pointers
\[
\mathtt{argv[0]},\ldots,\mathtt{argv[argc-1]}
\]
to the arguments of \MM.
\item
The function \Selectionfunction{selectmultimatchInit} is called 
\emph{after} the index is read and
\emph{before} the first \MMEM is processed.
The index can be accessed via the pointer \Datatype{virtualtree}.
This function can e.g.\ be used to initialize internal data 
structures defined by the user.
\item
The function \Selectionfunction{selectmultimatch} is applied to each match
referenced by \Datatype{multimatch}.
\begin{itemize}
\item
If \Selectionfunction{selectmultimatch} returns 
a value smaller than 0, then this is interpreted as an error,
and \MM exits with \index{error code} error code 1. 
\item
If \Selectionfunction{selectmultimatch} returns 0, then this is interpreted as 
\emph{false} and the match pointed to by \Datatype{multimatch}
is not selected. That is, it is discarded.  
\item
If \Selectionfunction{selectmultimatch} returns 1, then this is interpreted as 
\emph{true} and the match pointed to by \Datatype{multimatch}
is selected. That is, it is output by \MM according to 
the options given.
\end{itemize}
\item
The function \Selectionfunction{selectmultimatchWrap} is called \emph{after} the
last match has been processed. It can, e.g., be used to 
output internal data structures which store accumulated information 
about the \MMEM{s} found.
\end{itemize}
A selection function bundle must define at least the function
\Selectionfunction{selectmultimatch}. The other functions
can be omitted in the C-file declaring the selection function bundle.

The distribution of the programs described in this manual comes with a 
subdirectory \Filename{SELECT} containing
selection functions for diverse tasks:

\begin{description}
\item[formatmm.c]
provides a different output format for the output matches.
\item[selmulticontext.c]
selects a match whose first character begins with the first character 
of the given alphabet.
\item[sortmultimatches.c]
lexicographically sorts the multiple matches according to their
positions. To do so, the function
\Selectionfunction{selectmultimatchInit} first initializes a local table
to store the matches. \Selectionfunction{selectmultimatch}
stores the matches in this table. After the last match 
has been stored, the matches are sorted and output in
\Selectionfunction{selectmultimatchWrap}.
\end{description}
Note that in all these C-program files we use a macro
\Selectionfunction{DECLAREdynamicseqnumshared} which verifies that the
shared objects is called from the correct variant of \MM.

Besides these files, subdirectory
\Filename{SELECT} also contains a \Filename{makefile}, specifying how to
compile a shared object for the supported platforms.

\bibliographystyle{plain}
\bibliography{kurtz}

\end{document}
