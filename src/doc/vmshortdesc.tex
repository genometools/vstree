\documentclass[12pt]{article}
\usepackage{a4wide,times}
\author{Stefan Kurtz}
\title{\textbf{
A short description of Vmatch
}}
\date{\today}
\parskip5pt
\parindent0pt
\pagestyle{empty}
\begin{document}
\maketitle

Vmatch is a 
%DELETE versatile 
collection of software tools for 
%DELETE efficiently
solving large scale sequence matching tasks. It comes with a 
%REPLACE flexible user BY command line
command line interface and provides several 
%REPLACE unique BY unusual
unusual features.
%REPLACE achieves BY is designed to achieve
Vmatch is designed to achieve
efficiency by precomputing the sequence databases into an
index structure which is stored on different files.
The index structure 
%DELETE efficiently 
represents all substrings of the database sequences and allows,
unlike most other tools, to solve
%ADD many
many matching tasks in time 
independent of the size of the index. Different matching
tasks require different parts of the index, and only the 
required parts of the index are accessed during the matching process.

Vmatch allows to solve a multitude of different matching tasks 
over the index constructed in the preprocessing step. 
Every matching task is basically characterized by
the kind of matches sought,
additional constraints on the matches, and
the kind of postprocessing to be done with the matches.
In particular, Vmatch can compute the following kinds of matches:
maximal repeats, maximal substring matches, and complete matches.  
Maximal repeats and maximal substring matches can be taken as 
exact seeds and extended with two different strategies to obtain 
degenerate matches. The user can specify identity- and E-value-constraints
to select interesting matches. For each match found (exact or 
degenerate), the match and the corresponding alignment can be
shown. Alternatively, several postprocessing options are
available.

One of the 
%REPLACE unique BY unusual
unusual features of Vmatch is its alphabet
independency. It can process DNA and protein sequences,
but also sequences over 
%REPLACE any kind of alphabet, as specified by the user BY user defined alphabets
user defined alphabets.

The most common formats for input sequences (Fasta, Genbank, EMBL, and
SWISSPROT) are accepted by Vmatch. The user does not have to specify 
the input
%ADD if possible
format. It is automatically recognized, if possible. All input files can contain
%REPLACE an arbitrary number of BY many
many sequences. 
Gzipped compressed inputs are accepted.

The output of Vmatch 
%REPLACE is easy to parse BY can be parsed
can be parsed by other programs. 
Several options allow to customize the output.
New output formats can be incorporated without changing the 
program code. Certain subsets of matches can be selected by
user defined criteria.

\end{document}
