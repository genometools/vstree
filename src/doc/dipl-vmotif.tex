\documentclass[12pt,a4paper]{article}
\title{Vmotif: Efficient Searching of different classes of Biological Motifs
in large datasets}
\author{Diploma Thesis Topic; supervisor Stefan Kurtz}
\begin{document}
\maketitle
\section{Kinds of motifs}
\begin{enumerate}
\item
String motifs
\item
regular expressions 
\item
PSSMs
\end{enumerate}

\section{Operations on motifs}

\begin{enumerate}
\item
Determining the Significance of exact and approximate motif matches.
\item
If there are different ways to search a motif, then
determine the best strategy to search it. For example,
regular expression may be searched with automata and bitvectors.
There are many factors determining the choice of the strategy. These
often depend on the machine. Therefore the parameters should be
trained by running some tests for a representative class
of motifs.
\item
Strategy also depends on what parts of the index are 
available. The tables required are mapped in a lazy fashion,
i.e.\ only immediately before they are required.
\item
Searching online (exact and approximate, simultaneous search of 
different motifs).
\item
Searching offline (exact and approximate) based on Enhanced Suffix Array.
\end{enumerate}

\section{Algorithms to be implemented}

\begin{enumerate}
\item
exact motif search with Boyer-Moore
\item
automata search for regular expressions
\item
bitvector approach of agrep
\item
approximate motif search and restricted regular expression search of Myers
\end{enumerate}

\section{Kind of postprocessing the matches}
\begin{enumerate}
\item
Show motif match as alignment.
\item
Chaining of motif matches.
\item
Clustering of Matches (by content and positions)
\item
Masking of Matches
\item
Regions not covered by motif matches.
\item
Find a certain number of best motif matches, e.g.\ those matches 
with a minimum number of errors.
\end{enumerate}

\section{Features}
\begin{enumerate}
\item
Application to biological sequences
\item
Alphabet independency.
\item
Search direct and reverse complementary. 
\item
Allow to specify grouping of motifs. Find parts where a number of 
motifs of a group matches.
\item
For DNA Motifs and Protein sequence also search Motif in 6 reading frame
frames.
\item
Custom output via selection functions. Provide immediate postprocessing
of matches but also collection of matches and later combination 
of results.
\item
Portable to 64 bit platforms
\item
Use unique input format for motifs and supply parsers which 
produce this format from different other formats.
\item
Heavy automatic testing of all parts of the code.
\item
Follow coding guidelines described by Kurtz \& Gremme.
\end{enumerate}
\end{document}
