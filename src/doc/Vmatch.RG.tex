\documentclass[12pt]{article}
\usepackage{german}
\begin{document}
Angesichts des rasanten Wachstums der Sequenzdatenbanken l"a"st sich
eine Vielzahl von Sequenzanalyseaufgaben heute nicht mehr
mit den Verfahren erledigen, die in den vergangenen Jahrzehnten
entwickelt wurden. Ein Beispiel ist hier neben der Repeat-Analyse
der Genomvergleich von z.B.\ Mensch und Maus.
Anstatt die zu untersuchenden Sequenzen linear zu durchsuchen,
ist es in vielen F"allen sehr viel effizienter einen Index der
Sequenzen zu erstellen.  Hierzu haben wir das Konzept der erweiterten
Suffix-Arrays als Indexstrukur entwickelt \cite{ABO:KUR:OHL:2002} und 
im Werkzeug \emph{vmatch} effizient implementiert. Dieses erlaubt neben
der Suche verschiedener Arten von Repeats auch den
Vergleich kompletter Genome, die Suche nach exakten und approximativen 
Mustern, sowie das Maskieren und Clustern von DNA oder Proteinsequenzen,
Im DFG-Projekt "`Hybride Muster"' werden erweiterte Suffix-Arrays verwendet,
um komplexe Muster in RNA-Sequenzen zu suchen.
\emph{vmatch} ist f"ur die genannten Aufgaben oft um einige Gr"o"senordnungen
schneller als herk"ommliche Verfahren, die Sequenzen sequentiell durchsuchen.
Es hat bisher Anwendungen in den folgenden Bereichen gefunden:
\begin{enumerate}
\item
Beim Vergleich von Genomsequenzen und Proteinsequenzen
von Arabidopsis und Mais \cite{BRE:KUR:WAL:2002}.
\item
Bei der effizienten Ermittlung von eindeutigen Signatur-Sequenzen f"ur die
Erstellung von DNA-Assays \cite{FIT:GAR:KUC:KUR:MYE:OTT:SLE:VIT:ZEM:MCC:2002}.
\item
Bei der Erstellung von Gene-Indices \cite{COW:HAA:VIN:2002}.
\end{enumerate}
\bibliographystyle{plain}
\bibliography{defines,kurtz,biotools}

\end{document}
