\documentclass[12pt]{article}
\usepackage{fullpage}
\author{Stefan Kurtz}
\title{To Do List for Virtual Suffix Trees}
\begin{document}
\maketitle

\begin{enumerate}
\item

Add option -maxocc in vmerstat.

\item

Implement seed and extend algorithm in match-cluster to help Gunduz problems.

\item

What is electronic pcr?

\item

use the followign typedef

typedef union
{
  Sint is;
  float fs;
} Chainscoretype;

A global parameter determines if the score is an integer or if the score
is a float.

delete UNDEFCHAR from the code.

vstree2tex hat probleme mit zwei Tabellen bei option -bck.:
vstree2tex -ois -suf -bwt -bck -lcp -s bugseqs

Seqfault bei
vstree2tex -s -tis bugseqs

Allow range specification also for option -dnavsprot.

\item

Berechnung von Matches nur auf den drei Vorw"artsleserahmen oder auf
den drei R"uckw"artsleserahmen.

\item

Baue BLAT strategie ein: finde matches auf der gleichen Diagonale
und verbinde diese zu einem Match

\item

Option f"ur RARE matches.

\item

output option (laut Vorschlag von G Anders, 7.3.2005).

offset 1: Zeige Positionen beginnend mit 1
ignore # lines am Anfang
delimiter f"ur spaltenweise Ausgabe

siehe auch weitere Vorschl"age von G Anders, 7.3.2005.

\item

Ausgabe der Symbolverteilung in der Ausgabe von mkvtree
Kompaktere Darstellung aller Matches als cluster

\item

int x = 1;
        if(*(char *)&x == 1)
                printf("little-endian\n");
        else    printf("big-endian\n");

\item

mention the usefulness of the online option.

\item

extend a protein level match on the DNA sequence to become maximal?

\item

combine Supermaximal repeats together with option -p?

\item
Provides large collection of algorithms which are partly published,
but usually not available in a working tool.

\item
Add definition of clustering.

\item

-supermax and -i  gives wrong distribution of repeats.

\item
Several algorithms were published before, but Vmatch is the
first tool which implements them all.

\item

Algorithmic highlights of the paper

Look at competitor programs, according to E-mail from Mieg
20.10.2004.

\item
Add program to generate random sequences.

\item

nonredundant: see example for Tom Slezak
add examples from Newlink-Website.
All against-all script: siehe genallagainstall.pl
Mimic repfind and repselect
mimic maxmat3.x

\item

E-mail von Menzel: 
fragen, ob die Publikation akzeptiert ist.
Ja ist akzeptiert, publikation erscheint anfang April. Karin
bitten.

\item

Ebenso Martin Huh fragen.

\item

Ask Wei Zhu, see E-mail from 28. May, 2004.

\item

VMdb fertig machen.

\item

Script for hierarchical clustering: add this

\item

Now science:

I recently found some time to work on multimat, the program
implementing the first phase of MGA. It computes multiple matches
between the given genomes. To test multimat on a large dataset
I downloaded all sequences having the accession numbers you
listed in your last E-mail (Mouthfootdesease). Download was on Oct 1.
I got 126 files with a total of 8124 sequences. Their lengths
is in the range between 7733 and 8280.
Since construction of index mfdindex for all
the files took very long, there seemed to
be long common substrings. Indeed if I cluster the
sequenes with the following call of vmatch:

vmatch -l 7000 -seedlength 1000 -exdrop 2 -dbcluster 100 100 mfdindex

I obtain 110 clusters and 1 one singlet:

# 8123 elements out of 8124 (99.99%) are in clusters
# 1 elements out of 8124 (0.01%) are singlets

This means that mfdindex is the cluster index.
I clustered based on matches of length >= 7000 with seeds
of length >= 1000 and Xdrop value 2. The dbcluster parameters
100 100 specify that two sequences are clustered only if
a match covers the sequence completely. That is, the
sequences in each cluster are pairwise identical.

So it only makes sense to continue with the singlet and
one sequence from each cluster. I used the following call
to extract the 111 nonredundant sequences:

vmatch -l 7000 -seedlength 1000 -exdrop 2 -dbcluster 100 100 -v -nonredundant
nrset mfdindex

The non-redundant sequences are stored in the fasta file nrset.
Another call to mkvtree constructs the index nrset for the fasta file nrset.
Now lets looks for multimatches of length at least 10:

\item

Vmatch

The fastest, most memory-efficient way of comparing whole genomes!

The task of rapidly comparing entire chromosomes have typically required very
large computational resources. Such computing power and memory requirements are
orders of magnitude beyond what most users have on their workstations or
laptops. With the introduction of Vmatch this has dramatically changed! Very
large-scale sequence matching problems are now easily handled on systems found
on most researchers desktops or their portables.

The key to Vmatch's performance and small memory footprint is its incorporation
of an advanced data structure/indexing technique called Enhanced Suffix Arrays.
This variant of Suffix Trees has proven to be as powerful, with the advantage
of reducing both memory requirements and processing time. Careful
implementation of the data structures and algorithms have led to highly
optimized and robust operation. The standard version of Vmatch, compiled in
32-bit mode, allows the processing of sequences up to 400 million symbols. On
larger server class machines, Vmatch is available in 64-bit mode allowing the
processing of sequences gigabases in length.

Vmatch is extremely versatile enabling many different types of string matching
tasks including complete matches, maximal substring matches, maximal repeats,
branching tandem repeats and supermaximal repeats. Additionally Vmatch can take
the computed seeds and extend them to obtain long inexact matches and
alignments. Postprocessing operations enable the determination of substrings
not covered by a match, the masking of substrings covered by a match, and the
clustering of sets of sequences according to the found matches.

Vmatch already is being used in the following bioinformatics applications:

	� Comparing Two Complete Genomes 	� SNP Analysis
	� Genome Mapping 	� Sequence Clustering
	� Oligo Design 	� Sequence Masking
	� Designing Microarray Probes 	� Reducing Sequence Redundancy

\item
In this benchmarking exercise Vmatch, running on a Pentium IV 1.7 GHz PC,
outperforms the TimeLogic Decypher BLAST Solution by almost a full order of
magnitude!  (Under 3 hours versus 28 hours)

TimeLogic is a leading company in high speed bioinformatics computing. One of
its products is the DeCypher BLAST Solution, which is an innovative combination
of hardware engineering and software design to speed up BLAST searches. The
DeCypher BLAST Solution was shown to map 1.7 million 25-mer nucleic sequences
of proprietary SNP data to the human genome in 28 hours on a Sun Ultra Sparc
III server, which is 24X faster than a regular BLASTN search on a Linux cluster
of 30 Pentium III 1 GHz CPUs (see SNP Analysis graph midway down the DeCypher
BLAST Solution page).

For comparison, we generated 1.7 million simulated SNP data (file size about
85M) based on the lastest human genome assembly (Release 19.34b, downloaded
from Ensembl at ftp.ensembl.org/pub/human-19.34b/data/golden_path). It took
less than three hours for Vmatch to complete the entire mapping of these SNPs
onto the human genome on a Pentium IV 1.7 GHz CPU.


Summary of Vmatch Command Line Output

1. Make the virtual suffix tree (done in 79 seconds)
>mkvtree -dna -pl 12 -allout -v -db snp.fa
reading file "snp.fa"
maximal value for argument of option -pl is 12, recommended value is 11
total length of sequences: 44199999 (including 1699999 separators)
alphabet of size 5: acgtn
creating file "snp.fa.tis"
creating file "snp.fa.ois"
creating file "snp.fa.des"
creating file "snp.fa.sds"
creating file "snp.fa.lcp"
initializing data structures
sorting suffixes according to prefix of length 12
sorting all buckets
creating file "snp.fa.llv"
creating file "snp.fa.bck"
creating file "snp.fa.suf"
creating file "snp.fa.sti1"
creating file "snp.fa.bwt"
creating file "snp.fa.prj"
creating file "snp.fa.al1"
creating file "snp.fa.skp"
overall space peak: main=575.07 MB (13.64 bytes/symbol), secondary=123.27 MB

2. Map to the genome (done in 10,305 seconds)
> search_genome.csh snp.fa
reading file snp.fa
Mon Mar 22 17:41:05 CST 2004
Searching the genomic sequence /ensembl/19_34b_2/10.fa against snp.fa ...
Mon Mar 22 17:46:24 CST 2004
Searching the genomic sequence /ensembl/19_34b_2/10.masked.fa against snp.fa
...
Mon Mar 22 17:49:06 CST 2004
Searching the genomic sequence /ensembl/19_34b_2/11.fa against snp.fa ...
Mon Mar 22 17:54:23 CST 2004
Searching the genomic sequence /ensembl/19_34b_2/11.masked.fa against snp.fa
...
Mon Mar 22 17:56:59 CST 2004
...


Shell script source (search_genome.csh)
#!/bin/csh -f

set fout = "search.out"
rm -f $fout

foreach file (/ensembl/19_34b_2/[0-9XY]*.fa)
   date
   echo " Searching the genomic sequence $file against $1 ..."

   vmatch -q $file -l 25 -showdesc 0 -seedlength 12 -e 1 -d -p $1 >> $fout

   echo "" >> $fout
end 

\item

Case Study #2:  Oligonucleotide Probe Design and Validation

In this study Vmatch demonstrates its ability to detect potential
cross-hybridization, which may well prevent costly misinterpretation of
oligoarray experiments!


A critical problem in oligonucleotide probe design is to identify
unique regions in the target sequence set to avoid cross-hybridization with
non-target regions. Vmatch provides an efficient solution to this problem even
for very large target sequence sets. To illustrate this, we validate probe sets
used in a recent study by Kapranov et al to evaluate transcriptional activity
in human chromosomes 21 and 22 (http://www.sciencemag.org/
cgi/content/full/296/5569/916). Our validation consists of checking how many of
the probes have exact matches on human chromosome 1, thus potentially confusing
the original study by probe cross-hybridization to non-chromsome 21/22
transcripts. Full validation would merely involve running the same vmatch
commands on all human chromosomes.

The probe sets were downloaded from
http://www.affymetrix.com/transcriptome/sci_21_22_ paper/rawdata.html. The
13249s_del set interrogates every base pair of the Di George critical region
(362,901 bp) on the human chromosome 22. Using the clustering option of Vmatch
we show that 96.60% of the probes are non-redundant. Mapping to chromosome 1
revealed 1,632 distinct probes matching to 849,054 loci on that chromosome. The
second probe set (chipA) consists of selected probes in order to minimize
cross-hybridization, consisting of chromosome 21 and 22 sites with an average
35 bp gap between probes. This set is 99.57% non-redundant according to Vmatch
output, and it includes nine probes that perfectly match 272 distinct sites on
chromosome 1. Not all of these sites will be within the transcriptome. 

Command Line Output (13249s_del probe set)

1. Retrieve the probe sequences from the downlowded CDF file
>getProbe.pl 13249s_del.cdf > probe_13249.fa

Here is the retrieved 'probe_13249.fa' data.


2. Make the virtual suffix tree for the probe sequences
> mkvtree -dna -pl -allout -v -db probe_13249.fa
reading file "probe_13249.fa"
maximal value for argument of option -pl is 11, recommended value is 10
total length of sequences: 21013251 (including 808201 separators)
alphabet of size 5: acgtn
creating file "probe_13249.fa.tis"
creating file "probe_13249.fa.ois"
creating file "probe_13249.fa.des"
creating file "probe_13249.fa.sds"
creating file "probe_13249.fa.lcp"
initializing data structures
sorting suffixes according to prefix of length 10
sorting all buckets
creating file "probe_13249.fa.llv"
creating file "probe_13249.fa.bck"
creating file "probe_13249.fa.suf"
creating file "probe_13249.fa.sti1"
creating file "probe_13249.fa.bwt"
creating file "probe_13249.fa.prj"
creating file "probe_13249.fa.al1"
creating file "probe_13249.fa.skp"
overall space peak: main=166.01 MB (8.28 bytes/symbol), secondary=55.02 MB

3. Clustering
> vmatch -l 25 -dbcluster 100 100 dgcr -showdesc 0 probe_13249.fa
# args=-l 25 -showdesc 0 -dbcluster 100 100 dgcr
# /home/weizhu/dn/genechip/probe_13249.fa
# 9544 clusters
# 27486 elements out of 808202 (3.40%) are in clusters
# 780716 elements out of 808202 (96.60%) are singlets
# 6518 clusters of size 2
# 1236 clusters of size 3
# 696 clusters of size 4
# 316 clusters of size 5
# 284 clusters of size 6
# 200 clusters of size 7
# 90 clusters of size 8
# 52 clusters of size 9
# 30 clusters of size 10
# 34 clusters of size 11
# 22 clusters of size 12
# 18 clusters of size 13
# 14 clusters of size 14
# 4 clusters of size 15
# 8 clusters of size 17
# 6 clusters of size 18
# 2 clusters of size 19
# 2 clusters of size 20
# 2 clusters of size 22
# 2 clusters of size 25
# 2 clusters of size 28
# 2 clusters of size 29
# 2 clusters of size 31
# 2 clusters of size 33
0:
Cell417163_515_564
Cell417165_516_564
Cell417167_517_564
Cell417169_518_564
Cell417171_519_564
Cell417173_520_564
Cell417175_521_564
Cell417177_522_564
1:
Cell363779_523_504
Cell417179_523_564
Cell515493_730_674
Cell526259_773_686
2:
...

4. Search against the human chromosome 1
> vmatch -l 25 -showdesc 0 -v -q /ensembl/19_34b_2/1.fa probe_13249.fa
# args=-l 25 -showdesc 0 -v -q /TEMP/ensembl/19_34b_2/1.fa
# /home/weizhu/dn/genechip/probe_13249.fa
# matches are reported in the following way
# l(S) h(S) r(S) t l(Q) h(Q) r(Q) d e s i
# where:
# l = length
# h = sequence header
# r = relative position
# t = type (D=direct, P=palindromic)
# d = distance value (negative=hamming distance, 0=exact, positive=edit
# distance)
# e = E-value
# s = score value (negative=hamming score, positive=edit score)
# i = percent identity
# (S) = in Subject
# (Q) = in Query
# file=probe_13249.fa 36679661 21013251
# databaselength=21013251 (including 808201 separators)
# sequence lengths: minimal=25, maximal=25, average=25.00
# alphabet of size 5: acgtn
# probe_13249.fa.tis read
# probe_13249.fa.suf read
# probe_13249.fa.lcp read
# probe_13249.fa.sti1 read
# probe_13249.fa.bck read
# probe_13249.fa.des read
# probe_13249.fa.sds read
# probe_13249.fa.ssp read
# reading file "/TEMP/ensembl/19_34b_2/1.fa"
# querylength=246127941
# find direct substring matches against query
25 Cell648636_551_824 0 D 25 1 34697 0 2.58e+00 50 100.00
25 Cell648645_556_824 0 D 25 1 34698 0 2.58e+00 50 100.00
25 Cell648637_552_824 0 D 25 1 34698 0 2.58e+00 50 100.00
25 Cell187925_335_306 0 D 25 1 34698 0 2.58e+00 50 100.00
25 Cell187917_331_306 0 D 25 1 34698 0 2.58e+00 50 100.00
...



Command Line Output (chipA probe set)*

1. Retrieve the probe sequences from the downlowded CDF file
>getProbe.pl chipA.cdf > probe_chipA.fa

Here is the retrieved 'probe_chipA.fa' data.


3. Clustering
> vmatch -l 25 -dbcluster 100 100 chipA -showdesc 0 probe_chipA.fa
# args=-l 25 -dbcluster 100 100 chipA ?showdesc 0 /

\item

abbreviub still shows [NG] if a wild card is involved in a 
match. Use the union of the two characters instead.

\item

In vmatch -l 50 0 10 -i EcoliO157H7, the arguments 0 10 are ignored.

\item

Why are the branching tandem repeats not included in the repeats
computed by vmatch -l 50 0 0 EcoliO157H7?

\item
besides leftseq and rightseq, allow bothseq. This requires to compute 
a consensus of the two given sequences.

\item
Efficient decision procedure for deciding about similarities
between pairs of matches: use 

\item 
Implementiere Option -mumcand in Vmatch

\item
Bei Berechnung von paarweisen matches f"ur das clustern m"ussen nicht
immer alle vier paarungen gerechnet werden. Falls ein Match einen distanzwert
von 0 hat, dann reicht eine der beiden Matchinstanzen. Ordne die zu
vergleichden Sequenzpaare so, das die Paare 1/1 und 2/2 
zuerst kommen. Nehme zuerst die Paare mit "ahnlicher L"anger,
denn diese k"onnen "eher den gleichen Distanzwert erreichen.
Benutze evtl. Filtermethoden.

\item

checkgalign.x: option -ff sollte Fastafiles lesen k"onen, bw. eine
multiples fasta file um dann alle paarweisen alignments zu 
berechnen.

\item
folgendes Flag k"onnte das sein, was Du suchst.

#define SHOWALIGNMENTREVERSESUBJECTPOS (UintConst(1) << 8)

Daf"ur musst Du allerdings die showalignmentgeneric() Funktion benutzen.
Das gilt nur f"ur die obere Sequenz. Gordon fragen. 
21.2.2005: Will er erledigen.

\item

read pattern input from standard error. LATER

\item

indentation f"ur xml auf xmlindent umstellen. DONE. 2005-02-15

\item
add option -i to vmatchselect, as well as other options

\item
modify showdesc such that tab-separated output is given.

\item
Check alignments of complete matches, they look strange.

\item

generate compact environment a use this to compute approximate matches.
Verstehe ich nicht.

\item

modify alphabet such that there is an extra component 
wildcardoutchar, instead of having this stored in an array.

\item

XML-output for vmatch. DONE
Jan kann das so "andern wie er es m"ochte. Nachfragen.

\item

Why does the bwt change after the migration? Need other machine.

\item

maximum number of errors for pattern matching is only relevant
for computing edist substring matches. Also the greedy alignment function
have this restriction. It would be important to extend the greedy 
alignment function to work for arbitray many errors.

allow vmatch to process several different tasks of 
options in a row, for example, first mask internal repeats or
duplications, then mask repbase repeats, etc.

Sind die
> Rechte von  /vol/vstree/bin/SELECT/ mit Absicht so gesetzt ? Das ich
> als nicht Gruppenmitglied dort nicht schreiben kann ok, aber nicht lesen ...

> - wenig Sinn machen meiner Meinung nach auch die Rechte in
> /vol/vstree/share/doc/*.

Eine Sache ist mir im Manual aufgefallen. Der Parameter -online ist nicht
gl"ucklich erkl"art. Ich bin nach (zugegeben fl"uchtigem) lesen nicht
davon ausgegangen, dass man trotzdem eine (wenn auch kurze und schnelle)
Vorverarbeitung mit mkvtree durchlaufen muss.

- Du solltest evtl. andere Benutzer darauf aufmerksam machen, das sich das
  Interface der Selectionsfunktionen ge�ndert hat.

- Wenn man -online verwendet und f�r das query file der komplette Index
  vorhanden ist, wird trotzdem nur tis, des und sds gelesen und der 
  Index nochmal berechnet.

- Folgendes Problem ist noch offen:


BUG in smitwaterman: optimal local alignment of SmithWat Simple.fna
should be

CTTCGATCACTCGAC-CG
 !   !     !   !
CCTCGCTCACT-GACTCG

Add function to show Protein/DNA matches.

Allow vmatch to read rcm indexes and 6frame indexes. The rcm index
would for example allow to find tandem repats where one instance
is reverse complemented to the other.

store space blocks in a tree but not in a linear list. DONE.

manual: describe the different modes of showing an alignment: e.g.
        a =-symbol is put if there are symbols which map to the
        same number, due to an alphabet transformation.

for repeat-masking do not output the comment line 
in an extra header line, but add it to the description line.

Check the differences to reputer:

repfind.x -l 49 -e 3 -p -s /vol/biodata/DNA-mix/Grumbach/chntxx.seq

gives one more match.

integrate offline computation of complete matches with
errors into the program

Implement Simple Min-Cut Algorithms a la Stoer/Wagner

Evalues a la Karlin

/home/gremme/usr/arch/i686-pc-linux-gnu/vstree/include

Das bisherige sed skript l"oscht nur das Ende und l"ast

/home/gremme/usr/arch

add a function 

"ubrig

-----------------

Siehe auch Hinweise in Wei Email vom 1.3.2004.

Entwicklung eines Programms zu Design von bestimmen Fragmenten,
siehe Zusammenarbeit mit der Genomforschung in Bielefeld (Anke
Becker).

Compute Repeats with spacers according to the papers by Jens
Stoye.

Immitate the ARB programm, see mail from F.O. Gl"ockner.

for multimat: 
Compute a list of indices i for which asizes[i] is 1. This is the
constant list. Compute the leftchar for the constant list. If there
are different characters, then the match is always right maximal and
no extra checks are necessary. Otherwise, one has to compute the number 
of different characters.

consider a way to combine the virtual tree with position
information. That is solve queries like given an interval,
find if their is another match within a given sequence
range.

match a set of query sequences one by one.

add programs to add index tables to an existing index.

"readmulti.c", line 329: warning: statement not reached

Problems with cast from pointer to unsigned integer leads to problems
on tru64 machine:
cast from pointer to integer of different size
see bestmatch.c (in cmpBestmatch)   DONE

Eliminate all occurrences of random and initstate in the code.

Robert Homan merkt an.

Die von mapvirtualtreeifyoucan() ausgefuellte Virtualtree-Struktur
beinhaltet eine fehlerhafte Multiseq-Struktur. Die Werte von
vstree->multiseq.markpos.nextfreeUint und
vstree->multiseq.markpos.allocatedUint sind immer 0, auch wenn mehrere
Sequenzen vorhanden sind, der Wert von nextfreeUint sollte allerdings
vstree->multiseq.numofsequences-1 betragen. Das dazugehoerige Array
vstree->multiseq.markpos.spaceUint ist dagegen in Ordnung.

DONE, 21.2.2004

2. Ein Aufruf von checkspaceleak() gibt "space.c: line 393: This case is
not supposed to occur", wenn im Programm kein Speicher allokiert wurde.
Ich muss also z.b. in einem meiner Testprogramme einmal ein Byte
allokieren und es wieder freigeben, damit das Programm nicht einfach durch
checkspaceleak() beendet wird.

DONE, 21.2.2004

Es waere auch praktisch, wenn das Verzeichnis /vol/vstree/src/vstree/src
fuer die Gruppe lesbar gemacht werden koennte, dann koennte ich Dir auch
direkt Patches schicken.

implement a method with computes 
matches in a scanning approach over the query.
The last matches stretching over some 
region to the left of the scanning position are locally chained.
A chain exceeding some quality is then processed further.

sequenzposition bei -s option z"ahlen die gaps mit. Dadurch 
enstehen positionen, die of gar nicht mehr in der Sequenz
vorkommen.
(2) Gaps werden bei der Berechnung der Positionen nicht abgezogen.
Im Alignment passen die angezeigten positionen nicht: DONE, 3.9.2002

(1) bei palindromischen matches wird die index sequenz vorw"arts und
die query sequenz palindromisch dargestellt. hier m"usste auch
r"uckw"arts gez"ahlt werden...

remove vol every where in the src/bin scripts DONE

madvise does not work for OSX. Exclude it. DONE 3.1.2004

check for type problems with Optiontype. DONE

"light"-Version der libkurtz vorschlagen w"urde, siehe E-mail vom
4 Dez 2003. DONE.

In order to match 2 Draft genomes using vmatch, I need to get info
about the query file in the smae format as that generated by vseqinfo over the
index (query.seq.number its_length its_description). Is there is an option to
report this infos without constructing the index of the query. In the vmatch
verbose mode, I have found a concise description (e.g., total query length is xx
with y seperators).

Aenderungsw"unsche von Gordon, siehe E-mail vom 9.10.2003.

Write a paper on programs for questions like minimal unwords, minimal
unique subwords.

given an maximal error rate, how do we compute matches according
to this error rate. Problem, we do not know the length of the match
an can therefore not determine the number of errors allows.

problem: for a given number r, and a multiple sequence, find a match which 
occurs at least r times in one sequence, and not in all the other
sequences. Problem stated by Pevzner.

Die Werte fuer skptab sind um 1 hoeher als sie lt. manual
sein
sollten.  Hab nachgeschaut: im src-Code addierst Du 1
(warum?)

* virtman.pdf
Du definierst/erklaerst die bcktab nicht, erwaehnst sie
aber auf p61.

Nicht alle Optionen fuer vstree2tex sind aufgefuehrt.

Es muss heissen n >= 2^32 anstelle von <= auf p61, 6.
Satz.

ich f�nde folgende �nderung an den vstree-Makefiles sinnvoll:
das target clean sollte vielleicht auch die .pr-Dateien l�schen.

always try to recognize the the shortest unique prefix of an option.

die Introduction von mkvtreeman.pdf (so wie es unter /vol/vstree/share/doc
liegt) passt nicht, da sie sich auf Vmatch bezieht.

install -d does not work for SUN, because the directory
is not available.

map skipttab only if if matching regular expression is to be matched.

with option -v report the calls to the selection functions,
except for selectmatch.

specify function headers in select.h. The compiler will
reject declaration of functions if they do not match the
the type.

allow additional arguments after the first option of -selfun. These
are available in the argument vector and hence can be used in the 
selection function. Explain this by the low complexity selection where
the threshold is given as an extra argument.

argv[0] in selection function not well-defined.

add comments to FASTA files.

implement cluster strategies (according to their position) as well
as according to their string content.

check possible space leaks when return code of function is negative.
This could lead to problems when using the program in a GUI.

mga manual does not explain mga32.seqs und mga.128.seqs

read mohameds manual

allow option keepleft and keepright also in combination with 
dbmaskmatch.

allow the selection function to  submit a set of matches
at once and to process these as before. This would allow 
a much more flexible post processing of the matches.

if regular expression matches a description twice, then
this sequence is output twice.

implement option to select sequences which match the prefix of
description

Get mpneu and mgen as Genbank entries.
Get Ecoli as Embl entry.

make a list of error messages.

compute mum-candidates

introduce an option to deliver MUMmer3 output.

k"undige den Vertrag mit GLS.

evtl von stdin lesen

evtl gr"o"sere Platten anschaffen

modify output such that it resembles reputer

thanks for describing the problems you encountered. The segfaults should
not occur, and the error message are not very helpful.  I will
work on it to prevent users from combination of options
that are not allowed and to give them better feedback about what
options they should use. SEE E-mail of Pontus larsson.

with -help only show the basic options
add -helpextended for all options.

improve the HPUX functions in dlopen.h
improve their documentation
use quotes for our own header files, not < and >
remove clock.h from the automata files->Robert

add script to test vmigrate and vendian by applying it to 
an index twice it has to produce the same sequences.

clean up E-value computation. DONE

add option to vseqinfo to get the minimum length 
and the average length of the sequences.

vseqselect: If no option is used, then an error should be thrown.
If -minlength and -maxlength are chosen, then a message which sequence
was seelcted should be shown on stderr.

Improve error message by checking the external variable error number

\section{Bugs reported by Kirstin Weber}

\begin{enumerate}
\item
Es gibt nur bei vsubseqselect eine Fehlermeldung, wenn minlength
gr�sser ist als maxlength. Bei vseqselect gibt es sie nicht.
DONE.
\item
minimaler E-value 1.0e-300
Es wird keine Fehlermeldung ausgegeben, wenn dieser unterschritten wird.
DONE.
\item
vsubseqselect
\texttt{-minlength} und \texttt{-maxlength} gehen bis 256. 
Dies steht nicht auf Seite 7. 
DONE. This inconsistency in the program
has been removed. See Recent changes.
\item
vmigrate: keine Fehlermeldung, falls der Indexname falsch ist oder ganz fehlt
DONE.
\item
manchmal wird bei mkvtree 0 zur"uckgegeben, obwohl kein Fehler
aufgetreten ist. 
Ich brauch ein Beispiel, um den Fehler nachvollziehen zu k"onnen.
\item
Gibt es bei vmatchselect \texttt{-h} und \texttt{-hxdrop}? Die Optionen sind 
auf Seite 19 nicht aufgef�hrt, werden aber von einer anderen Option verlangt.
Ich brauch ein Beispiel, um den Fehler nachvollziehen zu k"onnen.
\item
Bldschirmphoto 2 zeigt eine f�r mich zuerst verwirrende
Fehlermeldung. Inzwischen glaube ich, dass nur meine
Festplatte voll war. Falls der Benutzer diese Fehlermeldung
bekommt, kann er vielleicht nichts damit anfangen.
Eventuell sollte man ihm die Meldung geben, dass nicht genug
Speicherplatz vorhanden ist. DONE
\end{enumerate}

\section{Correcting Errors and Inconsistencies}
\begin{enumerate}
\item
compile versions for the different platforms: e.g. clustalw,
mum.x est.x etc. make extra directory. 
\item
Generalize the script to generate goals in the makefiles.
\item
Make several versions of Multimat depending on the alphabet size.
\item
Add functionality for exact pattern matching using a suffix tree.
\item
What is the shortest substring over the DNA alphabet that does
not occur in the human genome?
\item
Add facility to compute at most a certain number of multimems, but not
more.
\item
minuniqueBFS.x seqfaults with hschr1
\item
update docu about multimatch -selfun
\item
mark an index if it is for 64-bit platforms. DONE
\item
in case the input string is read as plain text, sort
the alphabet according to the ascii order, as written in the
manual.
\item
Add readme to mga to explain what the files are for.
\item
implement program to map tables which works outside the debug environment.
\item
the maximal value for prefixlength should be larger. Currently it is at a 
maximum of 12.
\item
Implement algorithm of Dumas and Ninio.
\item
Computation of longest is still dependent on word size
\item
tandem does not trigger the postprocessing options.
for supermax it is probably the same: RepairTandemDBmask
\item
Allow to use vmatch -sort without -best.
\item
announce MGA to all reputer users (or a subset of these).
\item
E-values f"ur regul"are Ausdr"ucke.
\item
Sven Rahmann suggests:
In its current form, bcktab seems to waste a lot of space.
In fact, compare the end of the bucket for word w: bcktab[k(w)].q
Then the start of bucket for word w+1, bcktab[k(w)+1].p is 'almost'
at the same position, except when there are some special words
between the bucket boundaries, such as words with wildcards, with
text separators, etc. But bcktab[k(w)].q could be encoded with only
1 byte when interpreted as offset to the left of bcktab[k(w)+1].p
\item
Make the test independent from the index format, i.e. generate the
indexes in the current directory.
\item
Check why Iterqonl.sh does not always work.
\item
Why does the program find different matches as shown by Michael 
H\"ohl Examples?
\item
Some modules can be transferred from Vmatch to libsrc or into a library
virtuallib.a: fmumself.c, fsuper.c, markmat.c, opensel.c, xdropext.c
\item
allow multiple genbank/embl format DONE.
\end{enumerate}

\section{More Speed}
\begin{enumerate}
\item
add table to precompute power values for the E-value computation.
\item
the self comparison algorithm can be improved by
special consideration of the case where a branching node only
has two outgoing edges. These must be leaf edges. This can be deteced,
if in step ind, firstsucc = True, lcp[ind+1] < lcp[ind]. This holds e.g. for
42\% of the nodes in E.coli. Idea: First perform benchmarking to see where
the time goes. Use a first pass to mark \(i\) iff \(\SUF[i]<h\), 
\(\SUF[i+1]=h\), and \(\SUF[i+2]<h\) for some \(h\). That is those
\(h\)-intervals that are of width 2. Then \((h,\SUF[i],\SUF[i+1])\) is
a match.
\item
run the \(O(n\log n)\) time algorithm to generate an index 
if we expect long average lcp values. To decide this, apply a sampling 
approach to find that there are long consecutive suffixes.
\item
Choose the \texttt{queryspeedup} level according to the size of the data.
If both query and database are of similar size, then \(queryspeedup=2\).
If query is much smaller than database choose \(queryspeed=1\). If space
is at the premium, then choose \(queryspeed=0\). Other strategies may
be valid too.
\item
Add algorithm of Baeza-Yates as implemented by Jens Reeder.
\item
Add algorithm by Itoh as implemented by Janina Scholz.
\item
For exact matches, the conjecture is that contained matches are 
not produced if we skip isomorphic subtrees. 
\item
make the skip table relative to the current index: first tests 
to check this have been performed.
\item
Implement more efficient data structures to handle marked matches.
\item
only use intervals of the suffix array to compute the repeats:
For each of the intervals with suffixes of length at least \(d\) one
can perform independent processing.
\item
How can we incorporate more than one seed?
\item
pattern matching suffers from incorrect output.
\item
use halfmums as done in MUMmer2. The uniqueness is only guaranteed with
respect to one genome.
\end{enumerate}

\section{More Convenience}
\begin{enumerate}
\item
Add 64-bit version. DONE
\item
add breadth first algorithm for the masscmp problem.
\item
Add facility to select from an index those sequences that match a certain
expression in their description. This should be combined with selection for
the sequence content. This is a good project for Robert Homann. Everything
should be indexed based. DONE
\item
unify the file fcomplete: all function have the same type
\item
Delete the macros for copying structures.
\item
allow matching of two virtual trees against each other,
using some score function.
\item
manual needs some update, in particular the clustering part. DONE
\item
Produce an online-version of the manual. 
\item
make the search functions used in Vmatch into a library:
define an API (Application Programmer Interface)
\item
more infos with verbose option
\item
update the manual vstree.pdf
\item
vmatch
\begin{enumerate}
\item
give verbose output mimicing BLAST, maybe in a extra program.
\item
Selection of redundant sequences: the best in some sense should be shown.
\item
Add functions to translate DNA into proteins and then match against
Protein-DB. Volker would really like to see it. DONE
\item
implement subsequence option of the form \((i,j)\) also for complete matching.
\item
implement index search for complete approximate matching
\item
implement an online algorithm where possible: 
e.g. add edit matching for larger patterns
\item
add option postmerge: merge all repeats after computing them
\item
add option countwords
\item
return the result in XML format
\item
implement smaller evalues
\end{enumerate}
\item
mkvtree
\begin{enumerate}
\item
Define different global tasks and group the options accordingly. For example,
task repeats: options -l -exdrop etc
task querymatch: ...
task output:
\item
Implement readvirt in a more structured way: use the same code for the
different purposes.
\item
add function to compute derived tables on demand: recovering the index.
\end{enumerate}
\item
multimatch
\begin{enumerate}
\item
implement multimatch to find matches of \(k'\) sequences, where 
\(2\leq k'\leq k-1\).
\item
Instead of reporting all \((k+1)\) tuples, report the string that occurs
in all subsequences and for each sequence report the set of positions
where it occurs.
\item
determineMatchinfo liest sequenz von multimatch aus einer Datei.
Dann muss multimat.x auch eine args Zeile wegschreiben.
\item
Add specification of the syntax of regular expressions.
\item
make a small skip table by storing relative values.
\end{enumerate}
\end{enumerate}

\section{Conceptual Ideas}
\begin{enumerate}
\item
Can we use the $O(n\log n)$ algorithm to cluster local matches 
into chains. This would deliver long local alignments.
\item
When searching for maximal repeats, we do not have to consider subtrees
which are isomoprhic and not the maximal in their equivalence class.
This is due to the fact that they are not left maximal, since every
suffix begins with the same left character. Maybe use something like
the isotab to mark at which level the suffixes are isomorphic.
\item
How can we decide of a sequence of suffixes in an interval is
isomorphic and not maximal. Is there a way to see this from the
bwttab and the lcptab.
\item
when comparing genomes, find a substring \(s\) both of \(u\) and of \(v\)
such that \(s\) occurs exactly \(q\geq 1\) times in \(u\) and \(q\geq 1\) times
in \(v\). Only choosing the longest of these, should seperate the 
genomes such that a divide and conquer approach works.
\item
Add a section on Troubleshooting. Maybe call it Frequently Asked Questions.
Start with the message by Volker.
\begin{verbatim}
have you check the xdrop options to vmatch?  I misunderstand
something, or it is not working.  I match a pair of DNA sequences
with -l 10, say.  I find a couple of matches (identical >= 10-mers).
Then I add the option -exdrop 2, say.  All the previous matches
DISAPPEAR!

Can you check this (use any two DNA sequences)?
This came up in using XMKNR on DNA sequences - I have not been able
to generate sequence pairs allowing for many errors using any the
xdrop parameters.
\end{verbatim}
\end{enumerate}

\section{Papers to Read}
\begin{enumerate}
\item
Obtain the paper by Huson from WABI2001: Comparing Assemblies Using Fragments
and Mate-Pairs.
\item
Obtain the paper by Navarro and Raffinot from WAE2001: 
Compact DFA Representation for Fast Regular Expression Search
\item
Obtain the paper by Lancia from ESA2001: SNPs Problems, Complexity and 
Algorithms
\end{enumerate}
\end{document}
